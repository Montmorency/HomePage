\section{Introduction}
The Slater-Koster scheme \cite{slater54} provides a framework for estimating the numerical value of the large number of complicated 
integrals which appear in matrix elements between Bloch sums. By exploiting a local atom centered basis set, and
neglecting matrix elements between distant atoms, the number of these integrals can be reduced to a manageable form. 
The numerical value of the integrals can then be obtained via a fit to a known band structure or by direct computation. 

The Bloch sums in a local atomic basis set can be written:
%
\begin{equation}
\label{eq:tb}
\sum_{\R_{j}} e^{i\k\cdot(\R_{j}-\R_{i})}\times\int \psi^{*}_{n)(\r-\R_{i})H\psi_{m}(\r-\R_{j})dV.
\end{equation}
%
Eq.~\ref{eq:tb} yields the matrix elements arising in the tight binding basis, the eigenvalues can then be obtained
by solving the secular equation.

An alternative scheme is based on the empirical pseudopotential method. In this scheme the valence electrons are considered
to be moving in a smooth potential that can be defined with only a few fourier coefficients. The
coefficients of the potential in this scheme are obtained by a fit to experimental ($\k$-resolved) band structures. 
The emprical pseudopotential scheme has the advantage of providing directly the wavefunctions and eigenvalues 
whereas the tight-binding scheme's purpose is to avoid having to deal directly with the, potentially, 
complicated spatial structure of the wave functions.
%
\begin{equation}
\label{eq:emp}
\hat{H}_{EMP} = -\frac{1}{2}\nabla^{2} + \sum_{\G}V(\G)e^{-i\G\cdot\r}
\end{equation}
%
Both Hamiltonians commute with the translation operator $\hat{T}$ as they must for Bloch's theorem to apply.

The power of modern day computers is such that full {\it ab initio} methods are now common place for producing bandstructures.
The eigenvectors of the Kohn-Sham Hamiltonian can be computed directly:

\begin{equation}
\label{eq:kohnsham}
H = -\frac{1}{2}\nabla^{2} + V^{\rm ion}(\r-\R_{i}) + V^{\rm H}(\r-\R) + V^{\rm xc}(\r)
\end{equation}

These methods are hence reliant on some approximation for the exchange-correlation functional $V^{xc}$. Fortunately,
highly effective approximations to this potential exist. The most common functions are those which interpolate
the exchange-correlation potential of a free electron gas at low, intermediate, and high densities.

In this work we seek alternative parameterizations of the functional based on a combination of 
theoretical and empirical data. To accomplish this we decompose the tight binding and the empirical pseudopotential
into a sum of the exchange correlation potential, the kinetic energy, and then everything else, i.e.
the scattering of the valence electrons due to the atomic cores and the Hartree potential.
The procedure is then as follows: obtain the coefficients of Eq.~\ref{eq:emp} that best match experimental data.
Rotate these wavefunctions in Wannier functions suitable for evaluating the matrix elements in Eq.\ref{tb}.
Subtract the matrix elements of the ionic contribution, the hartree potential, and the kinetic energy leaving only
the exchange correlation part.

The most general function that can be considered is:
%
\begin{equation}
E_{ij} - [V^{H} + V^{ion} + \bra|-\frac{1}{2}\nabla^{2}|\ket] = E^{xc}_{ij} = \int w_{i}(\r-\R_{i})\Sigma(\r,\r';\omega)w_{j}(\r'-\R_{i})d\r d\r'
\end{equation}
%

However we will initially work with:
%
\begin{equation}
E_{ij} - [V^{H} + V^{ion} + \bra|-\frac{1}{2}\nabla^{2}|\ket] = E^{xc}_{ij} = \int w_{i}(\r-\R_{i})\frac{\delta E}{\delta[\rho(\r)]}w_{j}(\r-\R_{i}) d\r
\end{equation}
%

The $V^{xc} = \frac{\delta E \delta[\rho[r]]}$ is what we wish to fit. In the current formulation of the 
problem There is an indefinite number of ways to parameterize 
this function so that the required matrix elements are returned. In order to proceed further we need to restrict the
space of functions .

The two function spaces we explore in this work are $Vxc(\r)$ and using a Pade approximant. If we attempt a Pad\'e
fit via the direct inversion of the interpolating matrix we encounter severe problems with the condition number 
of the matrix making the accuracy and uniqueness of the fits difficult to quantify. We lose 
%
\begin{equation}
\label{eq:contfrac}
V^{xc}(\r) = \cfrac{1+}{a_{0}\rho(r) + \cfrac{b_{0}}{a_{1}*\rho(r)+b_{1}}...}
\end{equation}
%
In order for the function to be continuous in the density the poles of the function must lie somewhere in the 
complex plane off the positive real axis. 

The coefficients for Eq.~\ref{eq:contfrac} can be fit using an inference method.. 
The model requires that the coefficients provide an accurate fit 
across the experimental data set, as well as satisfying certain theoretical constraints in the high and low density limits.

First the fits of the potential in Eq.~\ref{eq:emp} will be made for each material. Upon obtaining the wave functions
they will be rotated into a Wannier representation. All the known {\it ab initio} components of the matrix elements
in Eq.~\ref{eq:tb} will be tabulated: i.e., the ionic, Hartree, and kinetic energy matrix elements. the remaining quantity,
$E^{xc}_{ij}$ will then become a target datum to fit the coefficients. For each material the number of target datum
generated will correspond to the number of local basis functions, $N_{\w}$, times the number of $\k$-points for which
we have optical data to hand, $N_{\k}$. The universality of the exchange correlation functional means it is greedy in
terms of the data that can be used to provide constraints on the coeffecients appearing in the expansion.

\section{Empirical Wannier Orbitals}
In Fig.~1 we present the empirical band structures.

\section{Perceptron Exchange Correlation Functionals}
\label{sec:bayesvxc}
The zeroth order requirement of the perceptron is that it can reproduce the existing parameterizations
of the exchange correlation functional. We can initialize the perceptron with these parameters when
we seek to optimize the experimental exchange correlation functional in Sec.~\ref{sec:bayesvxc}.
The quality of the fit that can be achieved is shown in Fig.~\ref{fig:1}. We are fitting to the 
Perdew-Zunger parameterization.

We now make some physically motivated arguments for our choice of parameterizing the
correlation and exchange parts of our functions. We argue that the important features of
the correlation functional are the local density and its most important supporting features
are auxiliary components of the density in $G$-space. The correlation part of the functional 
is sensitive to the dielectric screening in the material which is important over larger 
real space distances. The physical intuition for this is that intervening medium is required 
to screen the inter-electron repulsion. Furthermore the plasmon modes of the systems 
are determined by the density across the scale of a number of interatomic distances. 

Alternatively the exchange part of the functional has important features in the near real space
region where electrons approach each other. Hence the learning functional should be supported with
additional local real space components of the density. We note here that the empricial functional fitting procedure
would also be effective in atomic and molecular systems fitting to the optical excitation spectra starting
from an {\it ab initio} Hartree-Fock procedure.

For a given set of weights we can generate trial exchange correlation energies, 
measure the loss with respect to the empirical data, and then back propagate the 
weights and biases so that the perceptron is updated. In this manner the cost 
function is minimized and we can obtain our parameterization of the exchange correlation functional.

For the perceptron we only have the log-loss function as an indication of the quality 
of our final fit. 

\section{Results with Fitted Potentials}
  A couple {\it ab initio} band structures and some phonon spectra, along with a table of band gaps?
And a plot of the "new" exchange-correlation functionals to be compared with 
Fig.~1, PZ, PZ-NN, PZ-NN-G, PZ-NN-GX.

\section{Conclusions}
A Bayesian model for obtaining successive approximations to the exchange-correlation functional
has been presented. This approach provides a general metric for the effectiveness of exchange-correlation
functionals in terms of both generality and accuracy. The method is self-propagating in the
sense that more accurate experimental datasets would provide additional and more rigid constraints
on the weights and biases of the exchange-correlation perceptron.

The method is not entirely empirical, on the theoretical side
the eigenstates of the crystal and a rigorous mapping procedure 
through Wannier functions to a local atomic picture is necessary for 
fitting the matrix elements of the exchange correlation potential.
This is viewed as an extension to the Slater-Koster procedure
where the desired matrix elements are of the exchange
correlation functional rather than the entire crystal Hamiltonian.

The calculation of the kinetic energy, hartree potential, and, the ionic
pseudopotential, are all purely first principles quantities.

The formulation in terms of a single particle orbital and
target matrix elements makes it a well defined inference
problem and all the techniques of that field are applicable.
The data presented here gives parameterized functionals appropriate to
sp type systems in the diamond structure. However the procedure can be repeated for larger
datasets in arbitrary materials, work in this direction is underway.

