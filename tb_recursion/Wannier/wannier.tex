\chapter{Whistlestop Tour of Wannier Functions}
\label{chap:wannier}
\section{A Brief History of Local and Extended Waves}
Chemical intuition is built on a concept of local bonding. 
Yet the mathematics of Bl\"och waves requires a picture of an 
electron state extended throughout a crystal.

Pauling introduced the concept of linear combinations of atomic eigenfunctions
as forming hybrid orbitals which are local and directional in nature and
in many cases can be used to anticipate or explain the observed structure
of molecular compounds \cite{pauling28}. Pauling's publication was contemporaneous
with Felix Bl\"och's work on electronic wave functions in crystals \cite{bloch29}. The
way these two perspectives, that is, the local and the extended picturea
of electronic wave functions, are applied is the distinguishing feature 
of entire fields and sub-fields of physics and chemistry.

A definite mathematical transformation between the local formulation
and Bl\"och's formulation was obtained by Wannier. 
Wannier succeeded in obtaining a set of functions
from the set of Bl\"och waves which were localized in each unit cell
of a crystal and orthogonal to the wavefunctions in each of its
neighbouring cells \cite{wannier37, wannier62}.\index{Wannier,Bl\"och} 
This work, Wannier noted, was initiated by Eugene Wigner's suggestion that, 
``... there ought to be a way to reconcile the local and the band concept 
for electrons, and that such a reconciliation would probably be useful in 
understanding the spectra of insulators." 


However Wannier's transformation amounted to a gauge transformation
and did not result in a unique local description of the electronic states.
Koster proposed a variational procedure for the determination 
of Wannier functions \cite{koster53},
alongside a number of other early treatments of the problem \cite{winston54}.

The immediate practical consequence of Wannier's formalism was
it allowed for an effective computational treatment of impurities 
and donor and acceptor states in semiconductors \cite{slater49,kittel54} simply
by appealing to the existence of localized wave packets rather than calculating
them explicitly. This appeal to the existence of Wannier like orbitals is frequently made in
tight-binding formulations where they are not explicitly calculated but
matrix elements between them are inferred.

Lennard-Jones evolved a concept of equivalent 
orbitals \cite{lennardjones49a,lennardjones49b, hall50}
to determine the relation between electron-pairing theory of valency, 
the Pauling picture, and the molecular orbital theory by treating each 
electron as moving throughout the whole molecular 
framework. Like much of the early work on localized wave functions in molecules 
a group theory formulation of the problem was employed.
In Lennard-Jones' formulation a set of equivalent orbitals, 
when such a set exists, has the property that the effect of a symmetry 
operation only involves the interchange of equivalent orbitals.
Kimball has written from a similar `chemical perspective' 
on orbitals where combinations of spherical harmonics are taken to form
directed bonds \cite{kimball41} with a number of molecular examples.


Lowdin introduced a set of symmetrically orthogonalized wavefunctions which 
can be viewed as an intermediate formulation between hybrid orbitals 
and Wannier functions \cite{lowdin56}.

A unified treatment of each of these perspectives, molecular orbital,
Bloch states, Wannier functions and L\"owdin orbitals based on group theory 
has been made in Ref.~\cite{altmann58}. As well as a later review by Phillips\cite{phillips70}.


\section{MLWF: Maximally Localized Wannier Functions}
For an isolated set N of Bloch bands, in the Maximally Localised Wannier Functions 
(MLWF) formalism, the Wannier function is defined as:
%
\begin{equation}
|w_{n\R}\ket = \frac{V}{(2\pi)^{3}}\int_{\rm BZ} \left[ \sum_{m=1}^{N} U^{\k}_{mn}|\psi_{mk}\ket \right]e^{-i\k\cdot\R}d\k,
\end{equation}
%
where $U_{mn}^{\k}$ rotates the Bloch states at each $\k$-point.

The Maximally Localised Wannier Scheme proceeds by minimizing a spread functional:
%
\begin{equation}
\Omega = \sum_{n}^{N} \left[ \bra r^{2} \ket_{n} - |\bar{\r}|^{2} \right],
\end{equation}
%
a similar criterion for the determination of an optimal basis set for
molecular orbitals was given by Boys and Foster in Ref.~\cite{boys60b}.

Using Blount's relations it can be shown that 
the spread functional can be minimized by rotations of the overlap matrices defined as: 
%\cite{marzari97, mostofi08} 
%
\begin{equation}
M_{mn}^{\k,\bf{b}} = \bra u_{m\k}|u_{n,\k+\bf{b}} \ket.
\end{equation}
%

\subsection{Disentangling Bands}
When we aren't dealing with an isolated set of bands, i.e.
there are broad overlapping bands, e.g. an s-state that crosses in energy
across a group of $d$-like orbitals the procedure of Ref.~\cite{souza02}.
Two windows in energy are defined. An inner energy window of states are frozen 
and a wider set of states in an outer energy window (which contains the smaller
energy window) are projected down into a $N$ vector subspace:
%
\begin{equation}
|u^{\rm opt}_{n\k}\ket = \sum_{m \in N_{\rm win}^{\k}} U_{mn}^{{\rm dis}(\k)}|u_{m\k}\ket
\end{equation}
%

$U^{{\rm dis}(\k)}$ is a rectangular $N_{\rm win}^{\k} \times N$ matrix. 
The subspace is selected by minimising the quantity $\Omega_{I}$ with respect to $U^{{\rm dis}(\k)}$.
Where $\Omega_{I}$ is essentially the variation between neighbouring $\k$-points. 
%
\begin{equation}
\Omega_{I} = \frac{1}{N_{kp}} \sum_{\k, \mathbf{b}} \omega_{b} {\rm tr} \left[\hat{P}_{\k}\hat{Q}_{\k+\mathbf{b}}\right]
\end{equation}

$\hat{P}$ and $\hat{Q}$ are the projection operators: 
$\hat{P}_{\k}=\sum_{n=1}^{N}|u_{n\k}\ket \bra u_{n\k}|$ and $\hat{Q}_{\k} = 1-\hat{P}_{\k}$
minimizing this quantity corresponds to maximizing the overlap between rotated Bloch
states at neighbouring $\k$-points.

Once the optimal subspace has been determined the N vectors are then rotated in the same
was as an isolated set of bands to determine the Wannier function.

\subsection{Hamiltonian in the Wannier Gauge}
In the basis of Bloch wave vectors the Hamiltonian matrix is diagonal at each $\k$ point
with the diagonal element corresponding to the eigenvalue. To rotate to the 
Wannier gauge the optimal rotaion matrix can be used directly:

\begin{equation}
\label{eq:HWan}
H^{W}(\k) = (U^{\k})^{\dagger}(U^{{\rm dis}(\k)})^{\dagger}H(\k)U^{({\rm dis}(\k))}U^{(\k)}.
\end{equation}

$H^{W}$ refers to the Hamiltonian in the Wannier gauge. Eq.~\ref{eq:HWan} can then be contracted
over the $\k$ points to obtain:
%
\begin{equation}
\label{eq:Htb}
H_{nm}(\R) = \frac{1}{N_{\k}} \sum_{\k} e^{-i\k\cdot\R}H_{nm}^{W}(\k).
\end{equation}
%
$H_{nm}(\R)$ corresponds to the matrix elements between Wannier functions at different centers $\R$. 
The matrix elements are generated for the entire grid of supercell vectors $R$ that
are dual to the chosen kpoint. The inverse transform of \ref{eq:Htb} can be used to obtain a
dense $k$-point interpolation of the band structure Ref.~\cite{yates07}.

It is the matrix elements of Eq.~\ref{eq:Htb} that we wish to approximate using a tight binding scheme.
For a tight binding scheme to be valid these elements must decay relatively quickly 
as a function of $\R$ and transform according to the same symmetry operations as spherical harmonics
centered on the atomic sites. 

\subsection{Wanner Practical Procedure}
To carry out the procedure in practice an initial set of orbitals 
${g_{i}(\r)}$ is chosen and the matrix elements:
%
\begin{equation}
\label{eq:overlap}
A^{(\k)}_{mn} = \bra u_{m\k}|g_{n} \ket,
\end{equation}
%
are formed. $A^{\k}_{mn}$ is an $N^{(\k)}_{win} \times N$ matrix. 

A set of N functions at each $\k$ point in the Brillouin zone is given:

\begin{equation}
|\phi_{n\k} \ket = \sum_{m=1}^{N^{(\k)}_{win}} A^{(\k)_{mn}} |u_{m\k}\ket.
\end{equation}

The N functions are then L\"owdin orthogonalized to form an 
initial guess at the projected subspace:
%
\begin{equation}
|u^{\rm opt}_{n\k} \ket = \sum_{m=1}^{N^{\k}_{win}}(AS^{-1/2})_{mn}|u_{m\k}\ket
\end{equation}
%
The initial guess for $U^{\rm dis (\k)}$ is $\mathbf{A}\mathbf{S}^{-1/2}$ 
where $S_{mn}=(A^{\dagger}A)_{mn}$.

Though the minimization is robust a good guess at $g_{i}(\r)$ can speed the process.
For instance for $\alpha$-Fe the appropriate combination of initial states is a combination of 
$d_{xy}$, $d_{zx}$, and $d_{yz}$ orbitals along with 6 $sp^{3}d^{2}$ hybrid orbitals oriented along the 
cubic axes of the crystal. The final MLWFs bear a strong resemblance to these initial states.

The $U^{{\rm dis}(\k)}_{mn}$ matrix evolves from this guess in the course of the 
minimization so that the final rotation matrices at different k-points will have no 
obvious connection with the initial guess. Nor are the Wannier functions diagonal in the basis 
of the trial orbitals. If they were the trial orbitals themselves would be the Wannier functions. 

\section{Symmetry properties of the Wannier Functions}
The appendix of the Slater-Koster paper deals with the issue of whether L\"owdin type
orbitals, which are orthogonalized to their neighbours, transform according to the same
symmetry operations as the spherical harmonics.

\section{TB-NN}
Wannier functions are, it has been pointed out, just a gauge\footnote{What is a gauge?}. But
any gauge that can arrange Bloch orbitals which are spread across a crystal into a localized
packet of orthogonal orbitals will be of some considerable use in the local recursion techniques.
The essential point is that the local Hamiltonians, which the recursion techniques require,
must correspond formally to results obtained {it ab initio}. 
This correspondence must be quantitative and controllable and achieved in 
a way based on rigorous physical reasoning. 
This must be accomplished before we can move on to the real fun with the 
applications in Chapter \ref{eq:}.
Put simply it is essential that the houses are not built on sand! 
It is for this reason Chapter~\ref{chap:gw} contained what it did. The discussion of 
the previous work done on DFT, GW, and the Harris-Foulkes functionals is required to formulate
the local approach.

If they were it would be too easy for another to come along and tweak a few `parameters' to obtain 
new results. 

\section{Diamond Structure TB}
The choice of appropriate basis for a diamond tight binding model requires a little thinking.
The structure itself can be considered a Face Centered Cubic structure with two atoms per unit cell.
The lattice then has atoms at $a(p\bf{i} + q\bf{j} + r\bf{k})$ wherever $p+q+r$ is an even number
as well as at identical points shifted by $\frac{1}{2}a(i+j+k)$. Slater and Koster choose to proceed
by setting up $s,p_{x},p_{y},p_{z}$ orbitals on each of the two sites in a unit cell resulting in 8 
Bloch sums which need to be fit and including interactions between nearest and next nearest neighbours.
The nearest neighbours of one atom consist of a tetrahedral arrangement of atoms belonging to the 
second FCC lattice $a(1/2,1/2,1/2), a(1/2,-1/2,-1/2), a(-1/2,1/2,-1/2), a(-1/2,-1/2,1/2)$,
and 12 second nearest neighbours all located at $a(110)$ positions.

With the Wannier Functions we can choose a different basis. If we choose the origin of our coordinate
system as the center of a nearest neighbour bond between two atoms and set up a face centered cubic lattice
we can choose the eight sp3 wannier functions as a basis and infer the parameters
of their interactions with the 12 nearest neighbour cells of the type a(110). Then instead
of assigning the two center integrals $(ss\sigma)$ etc. to a particular atom, 
i.e. $(ss\sigma)_1$, we assign the two groups to the valence manifold and low lying conduction manifold 
$(ss\sigma)_{v}$. The fitting procedure then seek to fit the disposable constants to the W
annierized Hamiltonian matrix elements. Again a two center approximation can be used in the first pass
resulting in 8 neurons in the hidden layer, and then a second fit which makes use of 
all the disposable constants. The hope is that the MLWF basis will provide an optimal one in the
sense of the rapid spatial decay of the matrix elements between neighbouring unit cells 
(which for sp3 semi-conductors is pracitcally assured), and that the contributions of the three center
integrals will be negligible in this basis. Slater-Koster found that the two center approximation in their
basis was not a particularly good one and unable to interpolate the results obtained
from an orthogonalized plane wave calculation, however the atom centered spherical harmonic orbitals is likely 
to suffer from this deficiency, the requirement that neighbouring orbitals are orthogonal would lead to 
the formation of actual orbitals with non-zero components on neighbouring shells of atoms. 

Choosing the Wannier Basis results in orthogonal orbitals that remain well localized. This approach is similar to
that made by Hall \cite{hall52} who uses symmetric  combinations of directed orbitals
on adjacent pairs of atoms and constructs the bloch sums from these orbitals.
The second nearest neighbour s-interaction is found to have the opposite sign.

\section{Oh, What Larks}
There is little need to motivate the theoretical investigations of semiconductors
their technological relevance is undisputed. Nor is there a shortage of interesting
problems to consider surface reconstructions, donor acceptor levels, 
disorder \cite{hass40}, inversion layers.

