The asymptotic behaviour of  the recursion coefficients determines
how we terminate the continued fraction. The simplest case is where
the coefficients of the continued fraction $a_n,b_n$ each approach a constant.
In this case a square root terminator is appropriate (it is worthwhile convincing 
yourself of this). However in practice the coefficients do not approach constant 
values. They tend to oscillate and possibly have some weak asymptotic decay.
The literature contains most of the understanding of this behaviour. The 
asymptotic oscillations in the coefficients correspond to different types
of singularity in the spectrum e.g. band edges or van Hove singularities.

\subsection{The square root terminator}
This is the simplest and best place to start discussing 
terminators. For the constant chain we have the constant
coefficients $a$,$b$. In the absence of features like complex
band edges or singularities or band gaps this is a perfectly
serviceable scheme. The band of energiesis then a continuum in the range:
\begin{equation}
a-2b < E < a+2b.
\end{equation}

As summarized by Beer and Pettifor in Ref.~\cite{nato1984} if the 
bandwidth chosen is too small then the LDOS develops spurious peaks
at the band edges resulting in a loss of weight from the center of 
the band. These peaks have zero weight which then increases as they
move further away from the band. If a bandwidth that is too wide is 
chosen the "smoothness" of the DOS is lost and the spectrum tends
to a sum of delta functions.

\subsection{Tong's Method}
Tong et. al \cite{tong80} provide a useful interpretation of the asymptotic 
behaviour of the coefficients:

Take some level $i$ of the continued fraction:
\begin{equation}
g_{i} = \frac{1}{(E-a_{i})-b^{2}_{i+1}g_{i+1}}.
\end{equation}

Tong recognizes this as a special form of a M\"obius transformation:
%
\begin{equation}
z' = \frac{1}{\alpha - \beta z}
\end{equation}
%
Repeated application of the transform results in a motion of the point $z'$ through
the complex plane determined by two fixed points $z_{1},z_{2}$ which are the roots
of the equation:
%
\begin{equation}
\beta z^{2} - \alpha z + 1 = 0
\end{equation}
%
\begin{equation}
\frac{z-z_1}{z-z_2} = Q \frac{z'-z_1}{z'-z_2}
\end{equation}

where:
%
\begin{equation}
Q = \frac{-\alpha +\beta z_{2}}{\alpha + \beta z_{1}}
\end{equation}

$z_1=z_2$ motion is paraboli, if Q is complex and of absolute magnitude 1 it 
is elliptic and if Q is real the motion is hyperbolic (add figure here). 
In their analysis the hyperbolic case has consequences for localisation of states
since $|\psi_{E}(r)|^{2} ~ Im G(r)$ when the function crosses the real axis it 
may imply localization.

The graphical representation of the motion of g in the complex plane as the 
recursion coefficients are evolved is a useful one and it is further developed
in Turchi's method for spectra with Band gaps.

\subsection{Turchi's Method}

\subsection{Maximally Broken Time Reversal Symmetry}

\subsection{Akheizer}
In a purely mathematical formulation of the case of 
multiple bands we are looking at orthogonal polynomials
with a weight function defined along broken intervals. 

The general and early treatment of this was given by Akheizer
in terms of elliptic functions.


