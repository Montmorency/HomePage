\documentclass{article}
\usepackage[utf8x]{inputenc}
\usepackage{default}
\usepackage{graphicx}
\usepackage{amsmath}
\usepackage{hyperref}
\usepackage[sort&compress]{natbib}
\usepackage[a4paper,total={6in,9in}]{geometry}

\def\wp{\omega^\prime}
\def\w{\omega}
\def\wc{{\omega_{\rm C}}}
\def\ket{\rangle}
\def\bra{\langle}
\def\H{\hat{H}}
\def\P{\hat{P}_{\rm occ}}
\def\E{\varepsilon}
\def\vp{{v^\prime}}
\def\q{{\bf q}}
\def\s{\sigma}
\def\k{{\bf k}}
\def\qp{{\bf q^\prime}}
\def\G{{\bf G}}
\def\Gp{{\bf G^\prime}}
\def\rt{\tilde{r}}
\def\pt{\tilde{p}}
\def\r{{\bf r}}
\def\R{{\bf R}}
\def\rp{{\bf r^\prime}}
\def\rpp{{\bf r^{\prime\prime}}}
\def\rppp{{\bf r^{\prime\prime\prime}}}
\def\x{{\bf x}}
\def\mo{$\overline{1}$}
\def\mt{$\overline{2}$}
\def\S{\mathcal{S}}
\def\Gs{\mathcal{G}}
\def\v{\mathbf{v}}
\def\symm{\left\{\mathcal{S}|\mathbf{v}\right\}}

\usepackage{soul}
\usepackage{color}
\usepackage{subfig}

\definecolor{yellow}{rgb}{1,1,0}
\definecolor{lightblue}{rgb}{0.6,0.6,0.9}
\sethlcolor{yellow}

\def\g{\mathcal{g}}

\begin{document}
\title{Tight Binding Recursion and the GW approximation}
\author{Henry Lambert}
\date{\today}
\maketitle

\subsection{Application of Recursion Method to the Electron Gas}
	We now wish to apply the ideas discussed in section \. Specifically to 
calculate the exchange-correlation energy of an electron gas. Again it is 
no coincidence that the Pad\'e approximant is used. The coefficients
of a Pad\'e approximant can be calculated via a recursive algorithm
as a continued fraction \cite{vidbergserene} and, from the physical
intuition of the local environment, this is the appropriate functional form.

The idea would be to initialize spherically localized electrons in a box 
(where the box size determines the effective electron density)
then let them diffuse and calculate the contribution to the
single particle energy and the exchange correlation energy. 
If there is a clever choice of grids and co-ordinate systems this
should be easier to compute.

This is in line with the original work of Ceperley and Alder \cite{ceperley80} except 
instead of a diffusive Monte-Carlo approach we exploit the
recursion method to obtain analytic approximations to the energy.
In their case the trial function is a product
of two-body correlation factors times a Slater determinant of 
single-particle orbitals. The two-body correlation factors are
chosen such that they remove exactly the singularities in the local energy when two electrons
approach each other. The single particle states are plane waves
in the fluid phase: when the system is polarized, to allow for there only being
one spin per spatial state, the fermi wave vector is increased to allow
for twice as many spatial orbitals. 

In the crystal phase the orbitals are chosen to be Gaussians centered 
at body-centered cubic lattice sites. The simulations were typically 
for 38-248 electrons.

At each density the energy was extrapolated to an infinite number of particles
%
\begin{equation}
E(N) = E_{0} + E_{1}/N + E_{2}\Delta_{N}
\end{equation}

$E_{1}$ arises from the potential energy due to the correlation between a particle and its images
$E_{2}$ comes from the discrete nature of the Fermi sea for finite systems, $\Delta_{N}$ is the 
size dependence of an ideal Fermi system.

More details on the technique are given in an earlier version of the paper
\cite{ceperley78}. The BDJ wavefunction is written

\begin{equation}
\psi_{T}(R) = D(R) {\rm exp}(-\sum_{i<j} u(|\r_{i}-\r_{j}|))
\end{equation}
%
where D(R) is a Slater determinant of plane waves, separate slater determinants for
spin-up and spin-down particles in the unpolarized phase. The $u(r)$ is a pseudopotential
with a repulsive character and is intended to account for the particle correlation. 

\scriptsize
\bibliographystyle{unsrtnat}
\bibliography{refs}
\end{document}

