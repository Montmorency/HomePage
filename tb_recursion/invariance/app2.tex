\chapter{Rational Interpolation}
\label{app:ratint}
The rational interpolation using continued fractions algorithm
used in this thesis is described in Ref.~\cite{vidbergserene77}.
A generic function $C_N(z)$ is written as a continued fraction:
%
\begin{equation}
C_{N}(z) = \frac{a_{1}}{1+}\frac{a_{2}(z-z_{1})}{1+}...\frac{a_{N}(z-z_{N-1})}{1},
\end{equation}
%
where $z$ is the argument of the interpolating function
at the desired point, $z_{i}$ are the points the original function is sampled at,
and $a_{i}$ are the coefficients of the interpolating polynomial.
%
\begin{equation}
C_{N}(z_{i}) = u_{i}, \quad i=1,...N.
\end{equation}
%
the coefficients $a_{i}$ can be generated from the recursion relations:
%
\begin{equation}
a_{i} = g_{i}(z_{i}),\quad g_{i}(z_{i})=u_{i},\quad i=1,...,N.
\end{equation}
%
\begin{equation}
g_{p}(z) = \frac{g_{p-1}(z_{p-1}) - g_{p-1}(z)}{(z - z_{p-1})g_{p-1}(z)}, \quad p \geq 2.
\end{equation}
%
The value of the function can then be generated at the point $z$ using the relations:
%
\begin{equation}
C_{N}(z) = \frac{A_{N}(z)}{B_{N}(z)},
\end{equation}
%
where:
%
\begin{eqnarray}
A_{0}=0, \quad A_{1} = a_{1}, \quad B_{0}=B_{1}=1, \nonumber \\
A_{n+1}(z) = A_{n}(z) + (z-z_{n})a_{n+1}A_{n-1}(z), \nonumber \\
B_{n+1}(z) = B_{n}(z) + (z-z_{n})a_{n+1}B_{n-1}(z).
\end{eqnarray}


