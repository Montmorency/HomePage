\chapter{Off Diagonal Long Range Order}
\label{chap:odlro}
\section{Application to Long Range Off Diagonal Order}
The advantage of tight binding is it allows us to think about
the electronic structure of materials in a purely coordinate space representation.
Coordinate space is familiar in the sense we can discuss an electron as being
"here or there". This contrasts with the more familiar, at least when discussing crystals,
formalism of $\k$ space where all electronic eigenstates in a crystal are indexed by
their wavenumber. In $\k$ space an electron can be found anywhere in a crystal which
is somewhat frustrating. 

To justify the local representation we appeal unto the invariance theorem discussed 
by Heine and introduced in Chapter~\ref{chap:invariance}.

Having done all this work to get a local picture of the electronic structure of materials
it should be pointed out that many of the most interesting states of matter can't 
be understood in purely local terms. Among the most interesting state is the
superconducting state.

As an example Yang for ordered materials the local diagonal order present in crystalline materials,
represented by the matrix elements of $\rho_{1}$ (the single particle density matrix),
can result in the creation of macroscopic variables like the stress and strain.
What are the macroscopic analogs for off diagonal long range order?

C. N. Yang's work provides what I consider the most accessible treatment of long range
correlations in quantum systems and some of the important results and observations about
the nature of the crystalline and superconducting states are discussed in this chapter. 

\section{Yang's Off Diagonal Long Range Order}
Progessively higher density matrices are written with index n. 

Yang's insight is condensed in the following quotation.
%
\begin{quote}
It thus seems that in a macroscopic system, ODLRO can set in 
at $\rho_{n}$. The theorems of II indicate that
the reduced density matrix, to be called $\rho_{m}$, of lowest
order which has ODLRO must operate on a basic group that consists 
of an even number of fermions and any number of bosons. For liquid He II the basic
group is the He atom; for superconductors the basic group is a set of two electrons.
For $\rho_{m}$ the largest eigenvalue is of the order of N. It has an ODLRO
in the sense that in coordinate representation, when the unprimed coordinates are
microscopically close to a point $\x$, and the primed coordinates are microscopically
close to another point $\x'$. with $\x$ and $\x'$ macroscopically apart, $rho_{m}$
remains nonvanishing. For fixed unprimed coordinates microscopically
close together, the region of the primed coordinates where
$\rho_{m}$ remains nonvanishing is thus a "tube" with one 3-space dimension 
extending macroscopically. The volume of the region is 
\end{quote}
\begin{equation}
\Omega \times ({\rm microscopic\ dimension})^{m-1}
\end{equation}
%

The analogy Yang points out between Bloch's solution for wavefunctions 
in a crystal is also worth repeating in its entirety.

\begin{quote}
  The difference in the behavior of $R$ under changes in $\Phi$ for the cases 
with and without ODLRO is quite similar to the Bloch eigenvalue problem
in a periodic potential. Bloch showed that wave functions should be sought 
that changes by a phase factor $e^{i\Phi}$ for each lattice displacement. How does the 
wave function depend on $\Phi$? If the wave function remains finite between lattice points, 
the energy value and the wave function would be dependent on $\Phi$. If, however, 
the wave function becomes very small in a region between lattice points, 
caused by, e.g., a potential barrier, then the energy and the wave function 
would not be very much dependent on $\Phi$. For an infinite potential barrier in 
between the atoms, the wave function vanishes in the barrier, and the energy 
levels would be independent of $\Phi$ while the
wave function only picks up phase factors $e^{i\Phi}$ from one atom to the next.

The physical meaning of the effect of the presence of ODLRO on the 
phase condition (36) is the, ODLRO preserves the memory of phases over 
macroscopic distances. Also in this sense, one can interpret for the 
Bloch problem, the effect of the small nonvanishing interatomic 
value of the wave function on the C dependence of the energy: 
The nonvanishing interatomic value of the wave function preserves the 
memory of phases from one atom to the next.
\end{quote}

The wave functions dependence on $\k$ will be determined by how small 
the wave function amplitude becomes between unit cells.

This paper gives the first description of Bloch's function I have really understood. 
It also provides a sound mathematical basis for the pairing mechanism in BCS
superconductivity in the proof of Theorem 6 which Yang gives in the appendix.
%The R spacing argument can be applied to the electrons. The strip equivalent
%is microscopic (around the atom) but the spacing between phase locked electrons
%is of the order of the lattice constant.

In one way, Yang's formulation implies that the superconducting state
should be no more surprising than the formation of the solid crystalline state.

\section{Phonons}
The computation of the electron-phonon interactions in transition metals
can be computed using a tight binding formalism Ref.~\cite{varma79}.
In that work it is pointed out that d electrons form sharp resonances around
atoms but decay into plane wave states outside a localized atom so there
is a very delicate interplay between the electron being localized on an atom
and itinerant in the lattice. 
The hybridization of the d electrons with s-p electrons can also become 
significant.

\section{Antiresonance and Fano effects}
For a discussion of the Anderson model, AntiResonances, and the interaction of s and p channels
with d electrons see Ref.~\cite{terakura77}. Is an antiresonance a bit like what Anderson
talks about with electrons avoiding each other in space at a site? This is the super-exchange
mechanism he discusses?

The calculation of EPME (electron phonon matrix elements) and the computation 
of dielectric responses in a Wannier Basis are studied in \cite{terakura77}
and \cite{hank76}.

An early study at low temperature on Cu$_{2}$O\cite{white78}. An interesting study would
be to look at MgB2, cuprates, and pnictides to see all the studies which missed
the emergence of superconductivity at relatively high temperatures.

