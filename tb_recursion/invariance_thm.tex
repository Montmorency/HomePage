\section{Matching Green's Functions}
\label{sec:invariance}
  In Vol 35 Heine presents an extended discussion of the invariance
theorem. By way of introduction he points out an analogy between
the the local density of states written (for each spin direction):

\begin{eqnarray}
n(E) = \sum_{n}\delta(E-E_{n}) \\
n(E,\r) = \sum_{n} |\psi_{n}(\r)|^{2}\delta(E-E_{n})
\end{eqnarray}

and the density of black-body radiation $\rho(\omega,\r)$
with the substitution of the wavefunction with the electromagnetic
field and the energy level with a frequency $\omega$.

The local density of states is related in turn to the 
Green's function of a system:
\begin{equation}
n(E,\r) = -\frac{1}{\pi}{\rm Im} G(\r,\r',E)
\end{equation}

These tools should be more than sufficient to investigate what
we are interested in: namely cohesive properties, magnetic moments
and localized atomic forces in clusters. The math
should also be sufficient to devising schemes that allow for the correction
of artificial effects introduced into numerical work from the appearance
of spurious surfaces.

Heine's ambition turns on this:
\begin{quote}
We see now that the trick in the new formulation of quantum 
mechanics is not just to express measurable quantities in terms of 
G, {\emph but in terms of some appropriate small parts of G that can be solved 
for and computed separately from all the unwanted remainder of G.}
\end{quote}

This relies, again in analogy with black bodies, on the Green
function being independent of the size and shape
of the container (of electrons) and the boundary conditions.


First a change of variables is perform from r to r'' in 4.1 and r to r'' and r'
to r in 4.3
\begin{equation}
\label{eq:greentot}
[-\frac{1}{2}\nabla^{2}_{\r''} + V(\r'') - E]G(\r'',\r',E)= -\delta(\r''-\r')
\end{equation}

\begin{equation}
\label{eq:greenA}
[-\frac{1}{2}\nabla^{2}_{\r''} + V(\r'') - E]G_{A}(\r,\r'',E)= -\delta(\r''-\r')
\end{equation}

Now multiply Eq. \ref{eq:greentot} by $G_{A}$ and Eq.
\ref{eq:greenA} by $G$ Then integrate over r'' over region A using
Green's theorem:

\begin{equation}
\label{eq:greenthm}
\int\int\int(\phi\nabla^{2}\psi - \psi\nabla^{2}\phi)d^{3}\r 
= \int(\phi\nabla\psi-psi\nabla\phi)d{\mathbf{S}}
\end{equation}
 One then obtains

Subsuming the energy arguments we obtain an
expressions for G by integrating r'' over region A.

\begin{equation}
\label{eq:green1a}
G(\r,\r') = G_{A}(\r,\r') + \frac{1}{2} \int d\mathbf{S} 
[\frac{\partial G_{A}(\r,\r_{s})}{\partial n_{S}}G(\r_{S},\r')
- G_{A}(\r,\r_{S})\frac{\partial G(\r_{S},\r')}{\partial n_{S}} [\r, \r' \in A]]
\end{equation}

$\frac{\partial}{\partial n_{S}}$ denotes the normal component of grad across
the surface S and $\r_{S}$ is a point on S. (The grad term is a directional component
so can induce a change in sign!)

And in region B:
%
\begin{equation}
\label{eq:green1b}
G(\r,\r') = G_{B}(\r,\r') - \frac{1}{2} \int d\mathbf{S} 
[\frac{\partial G_{B}(\r,\r_{s})}{\partial n_{S}}G(\r_{S},\r')
- G_{B}(\r,\r_{S})\frac{\partial G(\r_{S},\r')}{\partial n_{S}} [\r, \r' \in B]]
\end{equation}
%

A further property has been used:
%
\begin{equation}
G(\r,\r',E) = G(\r',\r,E)
\end{equation}
%
which is often justified on the ground of a time reversal theorem
applying to the wave functions.

Finally consideration of r' and r'' in different regions is required.

With $\r''$ in B and $\r'$ in A Eq.~\ref{eq:greentot} becomes:
%
\begin{equation}
[-\frac{1}{2}\nabla^{2}_{\r''} + V(\r'') - E]G(\r'',\r')=0,
\end{equation}
%
the delta function is uniformally zero with those spatial constraints.

The next case restricts $\r$ in B:
%
\begin{equation}
[-\frac{1}{2}\nabla^{2}_{\r''} + V(\r'') - E]G_B(\r,\r'')= \delta(\r-\r''),
\end{equation}
%

A similar cross multiplying trick gives:
%
\begin{equation}
\label{eq:green1c}
G(\r,\r') = -\frac{1}{2} \int dS[\frac{\partial G_{B}(\r, \r_{S}}{\partial n_{S}} G(\r_{S},\r')
- G_{B}(\r, \r_{S})\frac{\partial G(\r_{S},\r'}{\partial n_{S}}] [\r in B, \r' in A],
\end{equation}
%
and a similar relation for [$\r$ in A and $\r'$ in B].

Eq.~\ref{green1a} and Eq.~\ref{green1b} give solutions for G but 
aren't very useful until the terms
$G(\r_S,\r)$ and $\frac{\partial G}{\partial n_{S}}$ are eliminated.

From Eq.~\ref{eq:green1a} we let $\r$ tend to the boundary so $\r = \r_{S}$
%
\begin{equation}
\label{eq:greensys1}
G(\r'',\r') = G_{A}(\r''_{S}, \r') + \frac{1}{2} \int d{S}
[\frac{\partial G_{A}(\r''_{S}, \r_{S})}{\partial n_{S}} G(\r_{S},\r) - 
G_{A}(\r''_{S},\r_{S})\frac{\partial{G(\r_{S}, \r')}{\partial n_{S}}}].
\end{equation}
%

The second relation comes from Eq.\ref{eq:green1c} with 
$\r$ in region B and letting $\r$ tend to $\r^{''}_{S}$.
%
\begin{equation}
\label{eq:greensys2}
G(r''_{S}, \r') = -\frac{1}{2}\int dS [\frac{\partial G_{B}(\r''_{S}, \r_{S})}{\partial n_{S}}
-G_{B}(\r''_{S},\r_{S})\frac{\partial G(\r_{S},\r')}{\partial n_{S}}]
\end{equation}
%

We now dump the cumbersome indices on the position attributes. These
were necessary to motivate the physical argument for how we have a Green's function
description in one region of space, a Green's function description in another, 
and the joined physical system should be some combination of these two descriptions
which match at the interface. The final step is to write Eq.~\ref{eq:greensys1} 
and Eq.~\ref{eq:greensys2}.

\def{\G}{\mathcal{G}}
\def{\g}{\mathcal{g}}

\begin{eqnarray}
\label{eq:operator1}
\g = \g_{A} + \frac{1}{2}\G'_{A}\g - \frac{1}{2}\G_{A}\g' \\
\g = -\frac{1}{2}\G'_{B}\g + \frac{1}{2}\G_{B}\g' \\
\end{eqnarray}

\begin{eqnarray}
\label{eq:operator2}
\g = [\G^{-1}_{A}(1-\frac{1}{2}\G'_{A}) + \G^{-1}_{B}(1+\frac{1}{2}\G'_{B})]^{-1} \G^{-1}_{A}\g_{A}
\g'= [(1-\frac{1}{2}\G'_{A})^{-1}\G_{A} + (1+\frac{1}{2}\G'_{B})^{-1}]^{-1}(1 - \frac{1}{2}\G'_{A})^{-1}2\g_{A}
\end{eqnarray}

$\g$, $\g_{A}$, $\g'$ are vectors in a space \{\r_{S}\} with components $G(\r_{S},\r')$,
$G_{A}(\r_{S},\r')$, $\partial G(\r_{S},\r')/\partial n_{S}$ with \r' fixed and
$\G_{A}$, $\G'_{A}$ are operators in the space with matrix elements
$G_(A)(\r''_{S}, \r_{S})$, $\partial G_{A}(\r''_{S}, \r_{S})/\partial n_{S}$.

Where the inverse in Eq.~\ref{eq:operator1, eq:operator2} 
doesn't exist there is a pole and this is the condition
for a discrete energy level of the Schrodinger equation 
for the combined A + B systems. 

Equations written in the form of \ref{eq:green1a} are of the type required for the invariance theorem namely:
\begin{equation}
G(\r,\r') = G_{A}(\r,\r') + {\rm boundary corrections}.
\end{equation}

These considerations provide a reasonable mathematical framework for 
investigating electronic structure from the point of view of the local environment.

\section{Recursion Method}
The principle of the recursion method, the details of the calculation will follow, is 
to obtain an efficient expression for a diagonal element of the Green's function.

\begin{equation}
G_{\chi\chi}(E) = \bra\chi|[E + i\delta -H]^{-1}|\chi\ket
\end{equation}

\begin{equation}
n_{\alpha l}(E) = -\frac{1}{\pi} {\rm Im}\bra\alpha l|[E + i\delta -H]^{-1}|\alpha l\ket
\end{equation}

$\chi$ can be made up of some linear combination of the local basis set $\phi_{\alpha l}$,
which, in turn, could be local atom centered orbitals, of bond orbitals, 
or even atomic diplacements to describe lattice vibrations. The method is thus
applicable to a wide range of problems in condensed matter.

\begin{equation}
b_{n+1}|u_{n+1}\ket = H |u_{n}\ket - a_{n}|u_{n}\ket - b_{n}|u_{n-1}\ket
\end{equation}

\subsection{Mathematical Origin}
The trick of the recursion method, if you like to think of things in terms of tricks, 
or its generating feature, if you prefer, is that from the outset the `greenian' is constricted
to the particular starting orbital $u_{0}$: 
%
\begin{equation}
G_{\alpha l, \alpha l}(E) = \bra u_{0}|[E-H]^{-1}|u_{0}\ket 
\end{equation}
%
As Heine puts it in his chapter:
%
\begin{quote}
Note that we do not want the whole of the inverse matrix $[E-H]^{-1}$, 
\emph{only one element}. and on this the whole method depends. In physics we do not
usually inquire about the whole of life, but about one particular matrix element (at a time) and we have chosen
u_{0} so that \ref{eq:greenian} gives us what we want.
\end{quote}

%
\begin{equation}
\bra u_{0}|[E-H]^{-1}|u_{0}\ket = \frac{{\rm det}|D_{1}|}{{\rm det}|D_{0}|}
\end{equation}
%

This can be equivalently written as:
\begin{equation}
\bra u_{0}|[E-H]^{-1}|u_{0}\ket = \frac{1}{\frac{{\rm det}|D_{0}|}{{\rm det}|D_{1}|}}
\end{equation}

Appeal is then made to Cauchy for the following expansion:

\begin{equation}
{\rm det}|D_{0}| = (E-a_{0}){\rm det}|D_{1}| - b^{2}_{1}{\rm det}|D_{2}|
\end{equation}

and when generalized to determinants for $|D_{n}|$, $|D_{n+1}|$, and $|D_{n+2}|$:

\begin{equation}
\frac{{\rm det}|D_{n}|}{{\rm det} D_{n+1}} = E-a_{n}) - \frac{b^{2}_{n+1}}{\frac{{\rm det}|D_{n+1}}{{\rm det}|D_n+2|}}
\end{equation}

\subsection{Integrated Quantities and Energy Difference between Two Substructures}
Some integrated quantities over the density of states of interest would be
the electron occupancy of a specific orbital:
%
\begin{equation}
N_{\alpha l}(E) = \int_{-\infty}^{E_{F}}E' n_{\alpha l}(E') dE'
\end{equation}

The total energy in the orbital is also convergent with increasing cluster size
#
\begin{equation}
U_{d} = \int_{-\infty}^{E_{F}}E'n_{\alpha l}(E') dE'
\end{equation}
#

Lets calculate energy difference of two structures A and B for the same transition metal
in fcc and hcp structures.
%
\begin{eqnarray}
\label{eq:Ua}
U_{A} = \int_{-\infty}^{E_{FA}} E n_{A}(E) dE
\end{eqnarray}

\begin{eqnarray}
\label{eq:Ub}
U_{B} = \int_{-\infty}^{E_{FB}} E n_{B}(E) dE
\end{eqnarray}

\begin{equation}
\label{eq:totaldos}
Z = \int_{-\infty}^{E_{FA}} n_{A}(E)dE = \int_{-\infty}^{E_{FB}}n_{B}(E)dE
\end{equation}

\begin{equation}
U_{A}-ZE_{FA}= \int_{-\infty}^{E_{FA}}(E-E_{FA})n_{A}(E)dE
\end{equation}

\begin{equation}
U_{B}-ZE_{FA}= \int_{-\infty}^{E_{FA}}(E-E_{FA})n_{B}(E)dE + \int_{E_{FB}}^{E_{FA}}n_{B}(E)dE
\end{equation}

If $n_{B}(E)$ is taken to be constant across that small energy range, the second term, in
Eq. results in a small squared term that can be dropped. Swapping $E_{FB}$ and $E_{FA}$ gives
a similar result. This justifies dropping the A index on the Fermi energy to get the
final expression:
%
\begin{equation}
U_{A}-U_{B} = \int_{-\infty}^{E_{F}}(E-E_{F})[n_{A}(E) -n_{B}(E)]dE.
\end{equation}
%
It is this expression used to separate the Lave phases of transition metals in  Ref.~\cite{haydock76}.

\section{Kelly}
An example from Kelly helps to make the shape of the Hamiltonian absolutely 
concrete.

\begin{quote}
As input data we require a convenient local representation of H. In simple crystals,
this is most easily given in terms of interactions with a unit cell and between neighboring cells.
For a fcc transition metal we have nine orbital (1s, 3p, 5d) per site, and a diagonal 9x9 matrix
for the self-energies of these orbitals. Twelve 9x9 matrices suffice to describe the interaction 
of any atom with its neighbors.
\end{quote}

And it is probably worth noting at this stage, for the practically minded,  
that the actual operations of $H|u_{n}\ket$ can be written as:
%
\begin{equation}
H|u_{n}\ket = \sum_{\alpha l \beta j}a_{n \alpha l}(\bra \beta j |H| \alpha l\ket)|\beta j\ket,
\end{equation}
%
and since we wish to exploit the sparsity of H we introduce an index $z$ which
runs over the neighbors of site l with which the orbitals interact:
\begin{equation}
H|u_{n}\ket = \sum_{\alpha l z}a_{n \alpha l}(\bra \beta_{z} j_{z} |H^{z}|\alpha l\ket)|\beta_{z} j_{z}\ket.
\end{equation}

There are no restrictions on the number of orbitals per site, or on the range of interactions that can be
incorporated, these only impact on the efficiency of the method and the validity of the original assumptions
about the dominance of the local atomic environment.

\section{Aside Green's Function Matching}
I would argue that every plane wave/pseudopotential calculation is in fact
an example of Green's function matching. Between atoms a basis of plane
waves describes interatomic bonding. In the vicinity of ions a pseudopotential
describes the way these plane waves are scattered at the atomic sites.
Indeed the logarithmic derivatives are required to match 
that of the atomic wavefunction at the cutoff radius for the pseudopotential.

\section{Free Electrons}

\begin{equation}
H = - \hbar^{2}/2m \nabla^{2}
\end{equation}


In spherical co-ordinates:
%
\begin{equation}
H = \frac{hbar^{2}}{2m}(d/dr^{2}  ) 
\end{equation}



\section{Future Work}
The chain model has been used in numerical work as a way of 
diagonalizing matrices in a computer 
C. Lanczos J. Res. Natl. Bur. Stand. 45, 255 (1950) and
by Chebyshev for approximating functions on an interval
P. L. Chebyshev, Bull. Acad. Imp. Sci. St. Petersbourg 1, 193 (1859)

What we wish to examine, is determined by u_{0}, this could be a combination
of neighbouring orbitals if we wish to determine the bond order, if it is the electronic
structure at a surface, the u_{0} should be a state at the surface.

These notes are intended to serve as an extremely abbreviated reference 
to Vol. 35 Solid State physics, and a basis for understanding the theory 
underlying the interactive recursion library.

The references in Vol. 35 demonstrate the scope of application of the method 
already achieved by 1980. A subset of those works have been referenced here. 
Haydock's conclusion is interesting in he gives prognosis for further work
along two lines: the extension to many-body Hamiltonians.

\begin{quote}
One of the outstanding probles is that of the Anderson model
of a disordered solid. Here the Hamiltonian is defined statistically and
one would like to transform to a corresponding statistically defined chain where
one could discuss the distributions of various physical quantities.
\end{quote}

\begin{quote}
Finally, the reader may have noticed that the approach of this chapter has been entirely directed
toward independent particle models. However, many-body Hamiltonians must be similarly 
transformable to chain models. Aside from a small amount of thought and continuing interest, 
nothing has been done about this and it remains a most interesting problem concerning
recursive solution of the Schrodinger equation.
\end{quote}

\bibliography
J Friedel Adv. Phys. 3, 446 (1954) #dilutealloys

Matching green's functions:
J. E. Inglesfield J. Phys C 4, L14 (1971)
B. Velicky and I. Bartos, J. Phys C 4, L104 (1971)
F. Garcia-Moliner and J Rubio J Phys C 2, 1789 (1969)
F. Garcia-Moliner and J Rubio Proc R. Soc. London, Ser. A 324, 257 (1971)

Non-hermitian matrices
R. Haydock and M.J. Kelly. J. PHys. C 8, L290
R. Haydock, J. Phys. A 7, 2120 (1974)

Application Recursion Method:
R. Haydock and M.J. Kelly, Surf. Sci. 38, 139 (1973) #d-orbitals #Fesurfaces #gallagherthesis
M. J. Kelly J. Phys. C 7 L157 (1974)
M. J. Kelly Surf. Sci. 43 587 (1974)

Perovskites:
I. Gyemant and M. J. Kelly, J. Phys. C 11, L193 (1978)

Chevrel Compounds:
Bullett Phys. Rev. Lett. 39, 664 (1977)

Lave Phases:
R. L. Johannes, R. Haydock, and Volker Heine Phys. Rev. Lett. 36, 372 1976 #lavephases

Anderson Localization
R. Haydock Philos. Mag. [Part] B 37, 97 (1978)
R. Haydock and A. Mookerejee, J. Phys. C 7, 3001 (1974)

Interfacial Energies
C.C. Pei Phys. Rev. B 18, 2583 (1978)

Self Consistency
J. J. Rehr and C.C. Pei, PRB 16 5506 (1977)
M. Mosteller and T. Kaplan,  PRB 19 552 (1979)

UV
R. K. C. McLean and R. Haydock, J. Phys. C 10, 1929 (1977)

