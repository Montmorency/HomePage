\chapter{Macroscopic Materials Properties and Bayesian Modelling}
\label{chap:bayes}
%There is another strange aspect to the properties of materials.
%That is that many properties of materials can be described by a scalar, 
%like Young's modulus, Poisson ratios, or small tensors to 
%describe the elasticity. 
%Electronic properties can be described using a single number 
%like the dielectric constant of a material.
%All these properties of systems are made up of many tiny degrees of freedom.
%yet at some point all these variables merge into macroscopic parameters.
%A satisfactory ab initio theory must capture this notion of {\it retraction on to} of
%the multidimensional wavefunction in a many electron material onto easily manipulated
%scalars or few indiced tensors.
%This paragraph needs research:
%Mention GAP models, empirical pseudopotentials, the lave phase stuff as well.
%Parameters of excitation and structure TB recursions can be updated as well.
%In addition many important theoretical developments came 
%from the context of attempting to explain a new set of experimental data.
%Bethe's calculation of the hyperfine? shift, the recursion work on the magnetic 
%moment of $FeAl_{3}$, I think Bardeen mentioned his work on superconductivity was
%initiated when they were trying, Shockley's work on the transistor. (A good source 
%for these sorts of anecdotal history can be found in Nobel prize banquet speeches.)

\section{}

\section{Bayesian Model Ranking}
A metric for determining what is meant by a minimal model
can be found in Bayesian probability theory. The Bayesian framework 
lets us define a metric that increases as the number of free 
parameters in the model is reduced, and the reproduction of 
target experimental data is increased, with a bound on optimal predictions.

The posterior on the parameters of the model can be assessed:
%
\begin{equation}
\label{eq:bayes}
P(\mathbf{w}|D, \mathcal{H}_{i}) = 
\frac{P(D|\mathbf{w}, \mathcal{H}_{i})P(\mathbf{w}|\mathcal{H})}{P(D|\mathcal{H}_{i})},
\end{equation}
%
where D, the target data, should constitute experimental data. 
Eq.~\ref{eq:bayes} MacKay summarises as:
%
\begin{equation}
{\rm Posterior} = \frac{{\rm Likelihood} \times {\rm Prior}}{{\rm Evidence}}
\end{equation}
%
\begin{equation}
\label{eq:bayesH}
P(\mathcal{H}_{i}|D) \propto P(D|\mathcal{H}_{i})P(\mathcal{H}_{i})
\end{equation}
%
\begin{equation}
\label{eq:bayesH}
P(D|\mathcal{H}_{i}) = \int P(D|\mathbf{w}, \mathcal{H}_{i})P(\mathbf{w}|\mathcal{H}_{i})d\mathbf{w}
\end{equation}
%
Models (hypotheses) of material systems can then be ranked according to Eq.~\ref{eq:bayesH},  

\section{Harmonic Approximations to the Free Energy and Entropy}
The crystal structure that is formed at a given temperature is that which
minimizes the Gibbs Free Energy of the system. Most electronic structure calculations
however are done at 0K. A systematic and computationally manageable way of obtaining
finite temperature energies, i.e. free energies is required.

Given a density of states . It is pleasing to report that the recursion method
is ideally suited to computing integrals of exactly that form, rapidly and accurately,
using quadrature methods. Given we now know how to compute electronic and 
vibrational density of states we are in the position to obtain, at no relative additional 
computational cost, vibrational entropies and Free Energies.

The work of Ref.~\cite{wheeler68} combined quadrature approaches
with the calculated moments of vibrational spectra to compute
thermodynamic quantites. For a comprehensive review
of entropic effects in metals and alloys see Ref.~\cite{fultz10}. 
These sources easily equip us with the expressions required. These expressions provide a starting
point for the inclusion of higher order terms in the vibrational spectra
the so called anharmonic effects which can become important at high temperatures.
