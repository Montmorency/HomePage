\chapter{Many Body Recursion}
\label{chap:manybodyrecursion}
\section{Introduction}
As mentioned at the close of Chapter.~\ref{chap:invariance} one of the 

The most direct extension of the recursion approach to many body Hamiltonians 
was undertaken In Ref.~\cite{annett94}. The concept of the local density of states (LDOS)
is extended to the projected density of transitions (PDOT) and it is argued singularities
in the PDoT correspond to thresholds for creating long lived elementary excitations 
in the many body system. In this chapter we demonstrate how to map a GW type calculation
on to a chain model. The analogy between an impurity or adsorbed atom interacting with
a constant chain and single particle states interacting via plasmon excitations is 
also pointed out the plasmon band serving to renormalize the single particle state.
The important calculations are then the position and bandwidth of the interaction
bands and the coupling of the single particle states to the interaction bands.
Efficient schemes for computing the positions of the interactions bands and 
their coupling to the single particle states (represented as Wannier functions)
are discussed.

Mapping the interacting problem to chain models means larger configurations can be studied
and the positions of band edges and defect states can be computed.

\section{Recursive GW}
We write down the many body Hamiltonian amenable in a manageable form:
%
\begin{equation}
\hat{H} = [-\frac{1}{2}\nabla^{2} + \tilde{v}_{\rm loc} + \hat{\Sigma}(\r,\r';\omega)]
\end{equation}
%
	$\Sigma(\r,\r';\omega)$ is a self energy operator, it is not local and 
has an additional argument that includes the energy.

The on-site self-interaction term can be computed as:
%
\begin{equation}
a_{0}(\omega) = \int\int\bra w_{i}(\r-\R_{i})|\hat{\Sigma}(\r,\r';\omega)|w_{i}(\r'-\R_{i})\ket d\r'd\r.
\end{equation}
%

The off diagonal elements of a tight binding Hamiltonian can be written as:
%
\begin{equation}
b_{ij}(w) = \int\int\bra w_{i}(\r-\R)|\hat{\Sigma}(\r,\r';\omega)|w_{j}(\r'-\R_{j})\ket d\r d\r'
\end{equation}
%
	Subsequent neighbour interactions would take a similar form
with the integration region, $\Omega_{j}$ being a Wigner cell around an 
atom with its Wannier center at $\R_j$. In order for the scheme to be sensible 
the contributions from subsequent shells must be negligible. This is likely 
to be the case so long as either the atomic orbitals $w$ and/or $\Sigma$ 
fall off quickly enough. Which, as will become clear when we discuss
the relationship of this scheme to the GW approximation, requires 
the Green's function or the screened Coulomb interaction to be sufficiently localized.

%
The self-consistency in the recursion method relies on the adjustment of the coefficients that
define the Green's function. If a suitable non interacting Hamiltonian is selected in the first
place this self-consistency can be treated as a perturbation problem.

In principle we can now compute any quantity we like that can be expressed
as a density of states, or an integral over the DOS, with the Many-Body Green's 
function just as in the non-interacting case discussed in Chap.~\ref{chap:invariance}.

%
%The link then has to come to Mori for random forces/coherent forces
%resulting in the emergence of the dielectric constant, plasmon frequency etc.
%
%The test case for this would be with Si, if 
%I can reproduce spectral functions with TB recursion that is a legitimate 
%proof of concept and would clarify the extent to which collective excitations,
%plasmons etc., can be incorporated in TB recursion. 
%


\section{The Self-Energy From Multiple Greenian Elements}
  A good first order approximation to $\Sigma$ comes from the GW approximation.
We will write matrix elements of this to see the connection between the
recursion method and the calculation of $\Sigma$. It should be come apparent that computing
the self-energy requires a large table of Greenian matrix elements between states localized
on nearby atoms, far away atoms, and the entire valence and conduction band spectrums. 
This table of Greenian elements could become excessively large very quickly and requires some
rigorous procedure to map it back into a manageable scale.
%
\begin{equation}
\label{eq:sigma}
\Sigma_{ij}(\omega) = \bra\phi_{i}|\int G(\omega-\omega')W(\omega')d\omega'|\phi_{j}\ket
\end{equation}
%
We now insert the identity operator into Eq.$\sigma$ strategically and at a number of places:
%
\begin{equation}
\label{eq:sigma}
\Sigma_{ij}(\omega) = \bra\phi_{i}|\int G(\omega')\sum_{n}|\phi_{n}\ket\bra\phi_{n}|(G(\omega')|\phi_{n}\ket\bra\phi_{n}|G(-\omega')v)d\omega'|\phi_{j}\ket
\end{equation}
%
\begin{equation}
\label{eq:sigma_onsite}%need to check the frequency polarization expression
\Sigma_{ii}(\omega) = \bra\phi_{i}|\int G(\omega')\sum_{n}|\phi_{n}\ket\bra\phi_{n}|(G(\omega')|\phi_{n}\ket\bra\phi_{n}|G(-\omega')v)d\omega'|\phi_{i}\ket
\end{equation}
%
The insertion of multiple identity operators makes what we need clear. What is required 
is to tabulate the Greenian's, $G_{ij}(\omega)$, between eigenstates of the system
the non-interacting system. In {\it ab initio} calculations this requirement 
gives rise to the "sum over states" problem. Although in principle the valence
and conduction states, at every wave vector, are known their calculation and storage
is tedious, and justified truncation of the sum is problematic.

For this scheme to be workable we again stress that the system interaction
must be sufficiently local such that $\Sigma_{ij}$ matrix elements can be
calculated. This corresponds to the mutual interaction of the electrons being 
short ranged. For a single particle tight binding model these off diagonal hopping
elements can be calculated and are fixed. It is this interaction we wish
to generalize slightly to allow for a component of dynamical interaction.
\section{The Project Density of Transitions}
The PDOT\cite{annett93, haydock00, haydock16} method is 
appropriate for calculating the polarizability and provides a direct physical
interpretation in a many electron system. 

Written in Hedin's scheme the polarizability 
takes the form of a term: $iGG$, in Feynman diagrams this is a bubble diagram. We need to calculate the propagator
of an electron from 1 to 2, and a hole from 2 to 1. Dimensionally, after a Fourier transform,
the polarizability has units of transition probability per unit energy. Where there are discrete
states there are discrete transitions, this would be the case for systems
that could be considered molecules, where there are continuous bands there are continuous
bands of transitions, in metallic crystals this would be the case.

Where the recursion method for the PDOS computes the final energy distribution
for a stationary state in the syste


I think Umari's operator for $|\phi\ket\bra\phi|$ would be a natural way of
approximating the necessary sums that are complimentary to the chosen
localized single particle states (both valence and low lying conduction).
We then have the continuum states as $|\phi\ket=e^{i\G\cdot\r}-<\G|w_{i}(\r-\R)>$,
with the occupied and low lying conduction manifold represented as Wannier 
functions and the $|\phi>$ states as orthogonalized plane waves.

	Although this is essentially just a rewriting of Hedin's equations
it has practical consequences. The present method eliminates
burdensome storage of what are in practice gigantic Green's function, 
screened Coulomb, and self-energy matrices. It also 
stabilizes the necessary numerical integrations and convolutions in frequency space. 
The quadrature methods used in the standard recursion theory remain applicable.

These savings are obtained at the price of a nested call stack of recursions which 
in principle must be infinite. This may seem troubling but we do intend to truncate
the level of recursion at some point and the depth of the recursive call stack should
equate to the number of terms in the self energy expansion that are required.

	Careful, justified, selection of the initial wave functions and tight binding parameters
is also required and these quantities (if they are not chosen to be analytic and approximate)
must be computed and stored. However work done on Wannier functions makes this a very
practical route. We will now embark on some numerical work in the hope that satisfactory 
results can be obtained by truncation of the recursive call stack at a manageable level. 

Finally the method requires computation of the initial stationary states of the system.

The electron addition operator:
%
\begin{equation}
[H, c^{\dagger}(t)] = -i\hbar \frac{d}{dt} c^{\dagger}(t).
\end{equation}
%
The operator will evolve into a superposition of a fraction of the 
possible transitions of the system. If a complete set of orthonormal $\{P_{\zeta}\}$ 
states is introduced the projected transitions can be written:
%
\begin{equation}
\psi_{\alpha} = \sum_{\zeta\eta} P_{\zeta}c^{\dagger}P_{\eta} \qquad \epsilon_{\alpha} = E_{\zeta} - E_{\eta}
\end{equation}
%
The PDOT is then:
%
\begin{equation}
\label{eq:ehpairs}
\mu(\epsilon) = \sum_{\alpha} \bra\psi_{\alpha}^{\dagger}|\psi_{\alpha}\ket \delta(\epsilon-\epsilon_{\alpha})
\end{equation}
%
The stationary states all have infinite lifetimes, and according to Haydock, \ref{eq:ehpairs} can
be interpreted as the probability distribution in energy for the decay of a bare electron into one dressed
with electron-hole pairs.

It is possible to formulate this in terms of discrete transitions for low energies, for high energies
the initial state is clearly going to diffuse outwards very far and result in the excitation of 
many oscillators throughout the crystal. Where the complete set of orthonormal states
hasn't been computed already (this refers to the sum over states problem) it is required to directly
calculate the PDOT. Haydock gives a prescription for this the time dependent behaviour of
an operator is given:
%
\begin{equation}
\label{eq:creationt}
c^{\dagger}(t) = e^{i\mathbf{L}t/\hbar} c^{\dagger}
\end{equation}
%

$\mathbf{L}$, the Liouvillian, is defined as a superoperator which
acts on an operator $\mathbf{x}$ as:
%
\begin{equation}
\mathbf{L} \mathbf{x} = [\mathbf{H}, \mathbf{x}]
\end{equation}
%
Eq.~\ref{eq:creationt} generates an expanding set of operators similar to what we encountered
in Chap.~\ref{chap:gw} at Eq.~\ref{eq:2pgreenfxn} where the field operators kept coupling
together as we expanded the many body interaction. Being able to truncate this expansion
rigorously is something we would like to do.

We would like to compare the mechanics discussed in the the project density of transitions
to Hedin's GW approximation, and provides a framework for computing the renormalization 
in terms of local orbitals and effective energy bands.

With the choices $u_{0}=c^{\dagger}/<c c^{\dagger}>^{\frac{1}{2}}$, and $u_{-1}=0$,
subsequent operators are:
%
\begin{equation}
\label{eq:pdotvectors}
\mathbf{u}_{n+1} = (\mathbf{L} \mathbf{u}_{n} - a_{n}-b_{n}\mathbf{u}_{n-1})/b_{n+1}
\end{equation}
%
\begin{equation}
a_{n} = \bra u^{\dagger}_{n} \mathbf{L} \mathbf{u}_{n} \ket
\end{equation}
%
\begin{equation}
b_{n+1} = \bra 
(\mathbf{L} \mathbf{u}_{n} - a_{n} \mathbf{u}_{n} - b_{n}u_{n-1})^{\dagger} 
(\mathbf{L} \mathbf{u}_{n} - a_{n} \mathbf{u}_{n} - b_{n}u_{n-1})
\ket^{\frac{1}{2}}
\end{equation}
%

Like conjugate gradients this recursion rarely terminates in practice 
due to loss of orthogonality of the vectors or because the subspace is 
too large to compute.

Where the charge density oscillations are on an interatomic scale of distance
we talk about single particle excitations, where they are across multiple 
interatomic spacings we speak of charge density oscillations and plasmons.

If the PDoT consists of a single band of transitions with a known width 
and energy,then the recurrence can be continued with constant matrix elements, $a_{n}$ and $b_{n}=b$
for $n>N$ where a and b are fit to the known band. 

For interacting bands it becomes more difficult to approximate a 
single effective energy band.

The termination of the recurrence is a very important point. 
We are interested in the excited state properties of an N+1 or N-1 electron system
from a local perspective. In Chapter.~\ref{chap:invariance} we worked hard to collect work
justifying this perspective. 

In a continued fraction expansion of the the PDoT more and more distance transitions are coupled
to the starting transition. In a real system the continued fraction will terminate at the surface,
however if the surface atom is changed by a small number of atomic positions it would
be somewhat surprising, and very interesting, if this had a strong effect on the electronic
properties around an atom deep in the cavity. Picking a value of anywhere from 30-60 moments 
away from a starting atom should help the reader to visualize this.

For silicon, the plasmon energy band occurs around 16 eV. This is very high compared
to the single particle excitations which occur around 1 eV. The long range $1/r$ nature of the Coulomb
interaction means electrons are coupled over large distances.
This coupling accounts for the high excitation energy: to get the
electron cloud moving collectively requires a relatively large amount of energy.

Experience with the GW approximation indicates an effective band model works very well
for sp semiconductors \cite{godbyneeds, hybertsenlouie, bergstressen}. 
A single plasmon mode approximates the long range density of transitions, and the local
atomic environment contributions to the PDoT can then be captured directly with the standard nearest
neighbour hopping and the inclusion of the lowest energy single particle transitions.

The situation for transition metals will be more complicated but again the local approach should yield 
a tractable framework. The important low energy transitions can all be included 
as explicit single particle states in the Hamiltonian (spin resolved states). 
The remaining particle-hole renormalization would then be accounted for 
via a set of effective plasmon energy bands. It would be of interest to
see how the number and position of these effective bands can change between systems.

The projected transitions can be written:
%
\begin{equation}
\mathbf{\psi}_{\alpha} = \sum_{n} \psi_{n}(\epsilon_{\alpha})\mathbf{u}_{n}
\end{equation}
%
the sum is over the basis ${\mathbf{u}_{n}}$ generated by Eq.~\ref{eq:pdotvectors}
with coefficients generated by:

\begin{equation}
\label{eq:pdotcoeff}
\psi_{n+1}(\epsilon) = ((\epsilon-a_{n})\psi_{n}(\epsilon) - b_{n}\psi_{n-1}(\epsilon))/b_{n+1}
\end{equation}

In his paper Haydock says equation \ref{eq:pdotcoeff} has two linearly independent solutions
which can be combined to satisfy the boundary conditions on $\psi_{N+1}$ or $\psi_{-1}$ or both.
Physically I understand this as choosing where the recurrence begins and ends not completely
clear on what it implies mathematically.

The $\mathbf{u}_{0}-\mathbf{u}_{0}$ element of the resolvent is:
%
\begin{eqnarray}
R(\epsilon)& = & 1/\epsilon-a_{0}-b_{1}\psi_{1}(\epsilon)/\psi_{0}(\epsilon) \\
  & = & 1/\epsilon-a_{0}-b^{2}_{1}\epsilon-a_{1}-b^{2}_{2}\epsilon...b_{n}^{2}/\epsilon-a_{n},
\end{eqnarray}
%
and finally the PDoT:
%
\begin{equation}
\mu(\epsilon) = -\bra \mathbf{c} \mathbf{c}^{\dagger}\ket {\rm Sing} \frac{R(\epsilon)}{2\pi i}
\end{equation}
%
The PDoT is the residue of the resolvent which contributes to integrals enclosing the 
real $\epsilon$-line and it is normalized to the projecting operator.

Calculating how much of the sum rule is exhausted will give a good indication
of the effectiveness of a single band approximation, or what is equivalent,
the truncation of the recursion at a suitable number of moments.

Finally a note from the perspective of plane waves and pseudopotentials. For
highly excited conduction band states. The Bloch periodic wave function is
essentially a plane wave or free electron state. This is equivalent to saying for
sufficiently high energy states the kinetic energy is the dominant term in the Hamiltonian.
For these plane wave states the Wannier transformation is a psinc function. These psincs
give a local representation of high energy plasmon modes.

\section{Appeal of A Local Formulation of Many Body Quantum Systems}
 A resonance or transition in k-space corresponds to an excitation 
of a crystalline eigenstate but the physical meaning of this is hard to access. Highly
localized states in $\k$-space correspond to relatively diffuse regions in real space. Where
transitions are formulated in terms of the local atomic environment the process becomes slightly
more intuitive. An electron creation/destruction operator $u_{0}$ places/removes an electron in the 
first available excited/occupied state with eigenvalue $a_{0}$ in a particular unit cell. 
In terms of energy the operator takes the system to the energy 
of the $E_{N\pm1}$ interacting system. 

Eq.~\ref{eq:pdotvectors} will contain the contributions of the localized
wave packets in a vector $u_{\alpha}$ that corresponds to the resonance 
$\epsilon_{\alpha}$. In an inverse photoemission experiment the $u_{\alpha}$ chain
will come from the commutator of an electron creation operator, for a photoemission
experiment the chain of vectors in the excitation will correspond to creating a hole
in the electron system. An electronic wave packet interacts with its neighbours becoming
``dressed" by the nearby excitations of electron hole pairs. 

In a photoemission experiment real electric currents will be moving through the crystal as well.
Only the $\k$-component of the localized wave packet, at a given energy, will move through the crystal as 
an approximate eigenstate and can pass through without scattering so much that it can't be detected.
When the wave packet reaches the surface it can continue propagating
or be scattered back into the crystal, if it is to continue propagating the appropriate matrix element
is between the plane wave of the appropriate wave vector and the localized wave packet along with some
surface potential.

Other processes mean the excited state can hop 
to its nearest neighbour leaving behind a hole. The recursive formula means
the coefficients of the continued fraction and the vectors are generated at
each step and the energy dependence of the interaction can be calculated 
in a closed form. The trade off is the recursion depth must be sufficient to
pick up all the relevant poles of the interaction. For collective modes
the recursion depth would have to proceed to an impracticable depth which
is why we must map these contributions onto effective energy bands, collective
variables, or plasmon modes (depending on how you prefer to call them). 
The technique demonstrates the mathematical and physical justification 
of the plasmon as a mathematical transformation or collective excitation.

The hole state and the excited state will also interact according 
to the Coulomb interaction, these are excitons, the radius of their (potentially) 
bound state being determined by the strength of the screened coulomb interaction 
and the shapes of the electron and hole wave function. 
However the PDoT we are considering at the moment is just a single electron creation operator.
To calculate the PDoT for the electron hole system would require a 
two particle Green's function operator with a new pole in the spectrum forming
at the energy of the electron hole bound state.

\section{Mori Memory Function}
\label{sec:morimemory}
  As noted in Ref.~\cite{annett94} there is a close connection between the PDOT
formalism and Mori's memory function. 

%Interestingly when working on the diffusion
%of hydrogen in a lattice I realised I would require a way of computing the
%inverse Laplace transform of the continued fraction generated 
%using Haydock's method. A quick search later and one of the first items to come up
%was a paper titled "A Continued-Fraction Representation of the Time-Correlation
%Functions" the first line of the abstract was, "A continued-fraction expansion of the Laplace
%transform of the time-correlation functions is obtained...". It is very nice when a problem
%arises where all you have to do is go to the shop and grab the solution off the shelf.

%Like many store bought items some element of self-assembly is required. In this
%section we wish to simplify Mori's analysis as much as possible so that we can directly 
%apply the mathematical machinery of translating a continued fraction representation of
%a diffusion problem into the time domain and extract measurable quantities.

The inner product, or matrix elements required to generate the coefficients 
$a_{n}$ and $b_{n}$ in Mori's approach are generalized to include a thermal averaging:
%\begin{equation}
%\label{eq:moriinner}
%(F, G^{*})= \frac{1}{\beta} \int_{0}^{\beta} \bra e^{\lambda H)F e^{-\lambda H}G^{*}\ket d\lambda
%\end{equation}
There is a further computational advantage here. Though the inner product in Eq.\ref{eq:moriinner}
looks complicated, in fact, it only has the effect of altering the coefficients entering
the hopping matrix. This means the laplace transform computed from the continued fraction,
and the quantum mechanical hopping parameters can be computed separately.

Mori's formula allows us to interpret the spectral function in two ways. The first is as
a purely phenomenological one of an electron in a material being excited and as it passes through
a crystal on the way to a detector being scattered by the characteristic scattering modes of the
system. The electron from its initial bound state can interact with the many body system via
the plasmon modes available to it.

The second is to interpret the probe as measuring a statistical distribution of the 
renormalized many-body electron states which have spectral weight across the full
interacting energy spectrum. 

The former is similar to the treatment of a molecules path
being modified by its collisions with atoms in what is termed Brownian motion. In the
case of a fermion in a crystal we are looking at its `Brownian motion' as it interacts
with bosons on its way to a detector. 

\section{Phase Transitions}
The parameters $\omega_{j}$ and $\Delta^{2}_{j}$ are writen in terms static
correlation functions some of them may have an anomalous temperature dependence.

\section{Applications }
GW type calculations are important for electron addition and removal 
experiments. We investigate the GW-recursion type calculations for 
impurities in Si. For data on this see Refs.~\cite{lehto78}.

%Perturbations:
%Haydock J. Phys. A Vol. 10, No. 4 (1977)
%Susceptibilities
%K. Terakura, J. Phys. C 11 469 (1978)
