\chapter{Many Body Recursion}
\section{Introduction}
Whether one approaches the problem of electron correlation from the
perspective of a chemist, physicist or a material scientist the 
essential work of this chapter is to build a formalism for `turning on'
the interactions between the electron wave packets as they have been 
determined in the single particle picture. 

We may mention a parallel but distant line of thinking
which sought to incorporate the effects of electron correlation
into the choice of basis set via the overlap matrices the work
of Adams, Anderson, and Boys is suggestive of this 
line \cite{adams62, weeks73}. 

The types of problems one may want to treat are discussed in \cite{slater49, kittel54}
and involve excitation energies and excitons. The interacting particles will also 
form the basis of the states required to discuss superconductivity 
in Chapter~\ref{chap:superconductivity}.

The work of Kelly \cite{kelly76} on comparing the electronic structure for different
topological configurations of Si demonstrate the importance of being able to combine
the local approach to calculations with the enhanced accuracy provided by
$GW$ type theories for describing electronic excitations. The extended nature
of structural defects must be incorporated into the calculations. The structures
in silicon i.e. the Polk-Boudreaux CRN is composed of 519 sites, the Connell-Temkin
model is composed of 238 sites.

\section{Recursive Many Body Formulations}
As mentioned at the close of Chapter.~\ref{chap:invariance} one of the potential
lines of extension for the recursion method is to move from a single particle 
description of the electronic structure of materials to a many body formalism.
We are also interested in carrying out tractable tight binding like calculations
which might incorporate more information about the local environment so that we
can compute our band gaps and cohesive energies as accurately as we may choose.

The most direct extension of the recursion approach to many body Hamiltonians 
was undertaken In Ref.~\cite{annett94}. The concept of the local density of states (LDOS)
is extended to the projected density of transitions (PDOT) and it is argued singularities
in the PDoT correspond to thresholds for creating long lived elementary excitations 
in the many body system. In this chapter we demonstrate how to map a GW type calculation
on to a chain model. 

The analogy between an impurity or adsorbed atom interacting with
a constant chain and single particle states interacting via plasmon excitations is 
also pointed out the plasmon band serving to renormalize the single particle state.
The important calculations are then the position and bandwidth of the interaction
bands and the coupling of the single particle states to the interaction bands.
Efficient schemes for computing the positions of the interactions bands and 
their coupling to the single particle states (represented as Wannier functions)
are discussed.

\section{Appeal of A Local Formulation of The Exchange-Correlation Functional}
A resonance or transition in k-space corresponds to an excitation 
of a crystalline eigenstate but the physical meaning of this is hard to access. Highly
localized states in $\k$-space correspond to relatively diffuse regions in real space. Where
transitions are formulated in terms of the local atomic environment the process becomes slightly
more intuitive. An electron creation/destruction operator $u_{0}$ places/removes an electron in the 
first available excited/occupied state with eigenvalue $a_{0}$ in a particular unit cell. 
In terms of energy the operator takes the system to the energy 
of the $E_{N\pm1}$ interacting system. 

Eq.~\ref{eq:pdotvectors} will contain the contributions of the localized
wave packets in a vector $u_{\alpha}$ that corresponds to the resonance 
$\epsilon_{\alpha}$. In an inverse photoemission experiment the $u_{\alpha}$ chain
will come from the commutator of an electron creation operator, for a photoemission
experiment the chain of vectors in the excitation will correspond to creating a hole
in the electron system. An electronic wave packet interacts with its neighbours becoming
``dressed" by the nearby excitations of electron hole pairs. 

In a photoemission experiment real electric currents will be moving through 
the crystal as well. Only the $\k$-component of the localized wave packet, 
at a given energy, will move through the crystal as 
an approximate eigenstate and can pass through without scattering 
so much that it can't be detected.
When the wave packet reaches the surface it can continue propagating
or be scattered back into the crystal, if it is to continue 
propagating the appropriate matrix element
is between the plane wave of the appropriate wave vector and 
the localized wave packet along with some surface potential.

The considerats at the close of Chap.~\ref{eq:}. Following the error 
analysis of the Harris-Foulkes functionals discussed in Chap.~\ref{chap:gw} 
we have good reason to hope that if our guess at the densities are very good, 
from the LDA, we can improve the $V_{\rm eff}$ term to include new physics.

\section{Fitting a GW-TB Model}
To begin we write down Eq.~\ref{eq:qpeq2} from Chap.~\ref{chap:gw}.
This is the Kohn-Sham Hamiltonian but with the many body contributions
encoded in the $\hat{Sigma}$ operator:
%
\begin{equation}
\hat{H} = [-\frac{1}{2}\nabla^{2} + \tilde{V}_{\rm loc} + \hat{\Sigma}(\r,\r';\omega)]
\end{equation}
%
$\Sigma(\r,\r';\omega)$ is a self energy operator. It has a few properties
that make it unwieldly to use. It is not local, so its operation requires an integral over
$\r'$ when it acts on a single particle state, and it has an additional argument 
energy argument $\omega$. 

To sidestep this we will seek to map the $\hat{Sigma}$ operator 
rigorously onto a local screening model with all the rotational and
three-center terms built in to it just as we did for the local Wannier 
tight binding model.

\section{Projections on to Local States}
The on-site self-interaction term can be computed as:
%
\begin{equation}
a_{0}(\omega) = \int\int\bra w_{i}(\r-\R_{i})|-\frac{1}{2}\nabla^{2} + \tilde{V}_{\rm loc} + \hat{\Sigma}(\r,\r';\omega)|w_{i}(\r'-\R_{i})\ket d\r'd\r.
\end{equation}
%
The off diagonal elements of the tight binding Hamiltonian can be written as:
%
\begin{equation}
b_{ij}(w) = \int\int\bra w_{i}(\r-\R)|\hat{\Sigma}(\r,\r';\omega)|w_{j}(\r'-\R_{j})\ket d\r d\r'
\end{equation}
