\chapter{Metallurgy}
\label{chap:metallurgy}
\section{Introduction}
In his textbook Cottrell noted that metallurgy would require a "theory... which explains
the properties of metallic solids and the explanation of the origin of this
structure from the structures of individual atoms." Similarly in Vol. 35
Heine noted "What we lack is a fundamental quantum theory of the cohesion
and structure of a wide diversity of solids." The previous chapters,
though hopefully enjoyable in their own right, were really just meant to bring
us to the stage where we can understand the properties of metallic solids
with the framework of a single computational tool.

This chapter will focus on the calculation of the electronic and mechanical
properties of transition metals and their alloys. The quantities of interest
are the elastic and chemical properties of dislocations, defects,
grain boundaries, and alloying.

In keeping with the unifying theme of the book the targeted quantities
will first be described and formulated in general terms and then mapped
on to a set of recursion calculations to understand the elastic and
electronic properties of a material from the perspective of the local
atomic environment.

\section{Elasticity}
Shearing, straining, Young's modulus, poisson ratio, elastic tensors these are the
stuff of every undergraduate material's mechanics course.
Ref.~\cite{terakura84} demonstrates how the recursion method
may be used to compute the elastic constants of a material.

The interaction of dislocations and defects reveals one of the most difficult problems
in theoretical materials science: the coupling of long range elastic fields
with local chemical changes. This coupling of length scales structure makes
reliable quantitative calculations difficult with variations in the electronic structure
inducing strains in the metal matrix which feed back into the electronic structure. 
It appears that one might require a theory with quantum accuracy on the scale of an 
interatomic bond that simultaneously takes into account the long range elastic 
field in its environment.

To proceed we will assume we can obtain a relaxed structure where all the atomic forces are
in equilibrium. We then follow the derivation of Ref.~\cite{freedman09} which uses the free energy
of the system at equilibrium to derive a tensor that describes the interaction of point defects 
with elastic fields. A few different ways of computing this tensor are then described \cite{nazarov16}.

The force method of computing the elastic dipole tensor lends itself 
immediately to the expressions introduced all the way back in 
Chapter~\ref{chap:invariance}, Eq.\ref{eq:tbforce}.

\section{Recursing through Dislocations}
The ability of the recursion method to quantify the local changes in interatomic bonding and
forces makes it a particularly useful tool for the study of dislocations. 

Understanding the interaction of interstitials and impurities with the motion of
dislocations through a crystal is a central goal of theoretical metallurgy with
with work on this subject beginning in the 50s \cite{cochardt55}.

The local density of states for atoms at a screw dislocation core have been computed by \cite{paidar81}. 
The change in the local bonding picture induced by a vacancy in Fe was examined in Ref.~\cite{ohta87}.

\section{Grain Boundaries}
The analysis of Read and Shockley in Ref.~\cite{read50} serves as a comprehensive
introduction to the field of interfaces in condensed matter. The definitive textbook
on the subject of interfaces in materials is Sutton and Baluffi Ref.~\cite{sutton95}.
The possibility of examining grain boundaries atomistically has interested
researchers since it became practical to perform full relaxations with pair
potentials Refs.~\cite{bristowe75,wolf83, rittner96, tschopp07}.
A hybrid TB-DFT model study is given by Paxton \cite{paxton96} on the structure of Nb, and Mo-Re.
At present for certain boundaries calculation may be performed entirely within
the framework of DFT \cite{momida13, du11, du12, mceniry18}.

\subsection{An Analytic Example}
For isotropic elasticity we can gain some useful practice in contour integration and arrive
at a closed set of expressions to obtain. These expressions are given in Ref.~\cite{sutton95}.

\section{Alloys}
\section{Random Alloys}
Lattice periodicity allows for the use of Bloch's theorem. The Fourier transform
is well defined and the eigenstates can be grouped according the their Bloch vector.
For alloys its quite likely there will not be an identifiable periodicity in 
the structure.
%
\begin{equation}
H_{nm} = e_{n}\delta_{nm} + V_{nm}
\end{equation}
%
Mookerjee and Haydock extended the recursion method to the case of random alloys.
The developement of the recursion method to handle random material systems is 
itself quite ingenious \cite{mookerjee, haydock74}. The Hamiltonian is extended
so that tight binding hops on a real lattice are made along with potential hops
on a "field" lattice. The field lattice can be cast in tridiagonal form to represent
the underlying statistical distribution of the random alloy. Different graphs are available
to represent rectangular random distributions, binary distributions, Gaussian distributions
etc. Fig.~\ref{fig:fieldsite.png} gives an example of the first few hops on a "field-site" 
lattice to obtain the averaged value of a Green's function. The hopping of the recursion method
is now coupling lattice sites and taking into account the underlying distribution of 
the underlying parameters.

Despite its elegance the number of paths now emanating from an initial orbital, both
for electrons to hop from orbital to orbital (lattice hops) and for the hops between 
statistical configurations (field hops) are considerable and simplifying approximations 
need to be found due to the difficulty of enumerating all the possible paths.

Early work on DOS of alloys \cite{cubiotti77}. 

\section{Conclusion}
In this chapter we have looked at a number of the important concepts
in metallurgy. The motion of dislocations, 
the structure of grain boundaries, and alloying. 

%Masuda-Jindo ISIJ 28, 843 (1988)
%Dilute Alloys
%J Friedel Adv. Phys. 3, 446 (1954)
%Non-Hermitian Matrices
%R. Haydock and M.J. Kelly. J. PHys. C 8, L290
%R. Haydock, J. Phys. A 7, 2120 (1974)
%Lave Phases:
%A. K. Sinha, Prog. Mater. Sci. 15, 81 (1972)
%Anderson Localization
%R. Haydock Philos. Mag. [Part] B 37, 97 (1978)
%R. Haydock and A. Mookerejee, J. Phys. C 7, 3001 (1974)
%UV Photoemission
%R. K. C. McLean and R. Haydock, J. Phys. C 10, 1929 (1977)
%https://materialscience.uoregon.edu/wp-content/uploads/2016/02/tightbind.pdf
%Self-Consistent Tight Binding Orbitals
%Philos. Mag [8] 35, 845 (1977) #
%Binary Alloys 
%R. L. Jacobs, J. Phys. F 3, 933 (1973)
%and stacking in La \cite{duthie77}.

%Atomic stress tensors:
%Sutton 88
%Nielsen O H and Martin R M 1983 Phys. Rev. Lett. 50 697
%Nielsen O H and Martin R M 1985a Phys. Rev. B 32 3780
%-                          1985b Phys. Rev. B 32 3792
