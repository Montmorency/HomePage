\chapter{Metallurgy}
\label{chap:metallurgy}
\section{Introduction}
Cottrell observed that metallurgy would require a ``theory... which explains
the properties of metallic solids and the explanation of the origin of this
structure from the structures of individual atoms." Similarly in Vol. 35
Heine noted "What we lack is a fundamental quantum theory of the cohesion
and structure of a wide diversity of solids." The previous chapters,
though hopefully enjoyable in their own right, were really just meant to bring
us to the stage where we can start using the principles of locality
and recursion methods to compute the structural 
and mechanical properties of extended solids.

This chapter will focus on the calculation of the electronic and mechanical
properties of transition metals and their alloys. The quantities of interest
being cohesive energies and elastic properties.

Perhaps the most interesting aspect of physical metallurgy is how
macroscopic parameters can be directly connected to the solution of 
the wave equation governing the atomic scale motion of the electrons.
The analysis of Wigner and Seitz in many ways initiated the field
of theoretical materials science with there calculation of the cohesive
energy and lattice constant of metallic sodium\cite{wigner33, wigner34}.


Subsequent work in this field extended Wigner and Seitz analysis
to the cohesive and elastic properties of monovalent
metals like sodium, lithium, and copper \cite{fuchs36}. The quantum theory was
also used to explain the appearance of different phases in alloys in terms 
of Brillouin zones \cite{bethe29, bouckhaert36, owen33, jones34}. 

Born's work provides the standard treatment of the stability of lattices 
to short and long wavelength
vibrations considering general forms for the interatomic potentials\cite{born40,born42},
with other workers again making extensions of the theory to particular
lattice types \cite{power41, nabarro52}.

In keeping with the unifying theme of the book the targeted quantities
will first be described and formulated in general terms and then mapped
on to a set of recursion calculations to compute the cohesive and elastic
properties of a material from the perspective of the local atomic environment.

\section{Crystal Structure}
Step number uno in the field of theoretical metallurgy is a theory that correctly
describes the crystal structures formed by different pure metals. 
Early considerations based on orbital hybridisation and geometry
were performed by Altmann, Coulson, and Hume-Rothery who treated 
the problem in Ref.~\cite{altmann57}. One of the earliest triumphs 
of the recursion method the sorting of the Lave 
Phases \cite{haydocklaves75,johannes76}. Density functional
studies of the crystal phases \cite{chan86,paxton97}

The ability of the tight binding framework
to provide an explanatory model for trends in homologous materials is
well demonstrated by the work on the crystal structure along the transition
metal series \cite{pettifor70}, the Chevrel Compounds of Molybdenum Chalcogenides Ref.~\cite{bullet77}.
and the structures of the compounds in the Lave phases in Ref.~\cite{johannes76},

\section{Vibrations}
Probing the local forces in a material is possible with neutron spectroscopy.
Interplanar force constants can be accessed directly which gives experimental
insight into \cite{varma79a, varma79b}.

\section{Harmonic Approximations to the Free Energy and Entropy}
As discussed in Chapter~\ref{chap:invariance} Sec.~\ref{sec:}. 
The work of Ref.~\cite{wheeler68} combined quadrature approaches
with the calculated moments of vibrational spectra to compute
thermodynamic euantites. Brent Fultz is a comprehensive review
of entropic effects in metals and alloys Ref.~\cite{fultz10} equip us
with the expressions required. These expressions provide a starting
point for the inclusion of higher order terms in the vibrational spectra
the so called anharmonic effects which can become important at high temperatures.

\section{Stress, Strain, and Elasticity}
Shearing, straining, Young's modulus, poisson ratio, elastic tensors these are the
stuff of every undergraduate material's mechanics course.
Imagine you have some tensor $\epsilon$ and you apply a virtual pure strain to a collection atoms.
The TBB Model \cite{nielsen83, sutton88} gives us a means of defining and calculation 
a local in the sense of a per atom stress tensor. If $I$ and $J$ 
are cartesian components $x,y,z$ to first order in the energy:
%
\begin{equation}
\delta E^(1) = \sum_{I} \sum_{(ij)}\Omega^{i}\sigma^{i}_{IJ}\epsilon_{IJ}
\end{equation}
%
The contribution to the IJ component of the stress tensor is:
%
\begin{equation}
\label{eq:atomstress}
\sigma^{i}_{IJ} = \frac{1}{\Omega^{i}}\sum_{j\neq i}\sum_{\alpha\beta}G(\epsilon)_{i\alpha j\beta}\frac{\partial H_{j\beta i\alpha}}{\partial R_{jI}}(R_{jJ}-R_{iJ})
\end{equation}
%
The pair contributions can be added to obtain the full stress tensor. 
But if we are only interested in the electronic contributions
Eq.~\ref{eq:atomstress} is already good enough. For an application of the 
see Ref.~\cite{ohta90}

As usual we need to know the $R$ dependence of our
integrals, both hopping and on-site, which we went to the trouble 
of learning in Chapter~\ref{chap:wannier}.

Ref.~\cite{terakura84} demonstrates how the recursion method
may be used to compute the elastic constants of a material.

The interaction of dislocations and defects reveals one of the most difficult problems
in theoretical materials science: the coupling of long range elastic fields
with local chemical changes. This coupling of length scales structure makes
reliable quantitative calculations difficult with variations in the electronic structure
inducing strains in the metal matrix which feed back into the electronic structure. 
It appears that one might require a theory with quantum accuracy on the scale of an 
interatomic bond that simultaneously takes into account the long range elastic 
field in its environment.

To proceed we will assume we can obtain a relaxed structure where all the atomic forces are
in equilibrium. We then follow the derivation of Ref.~\cite{freedman09} which uses the free energy
of the system at equilibrium to derive a tensor that describes the interaction of point defects 
with elastic fields. A few different ways of computing this tensor are then described \cite{nazarov16}.

The force method of computing the elastic dipole tensor lends itself 
immediately to the expressions introduced all the way back in 
Chapter~\ref{chap:invariance}, Eq.\ref{eq:tbforce}. \cite{paxton90}.

\section{Recursing through Defects and Dislocations}
The ability of the recursion method to quantify the local changes in interatomic bonding and
forces makes it a particularly useful tool for the study of dislocations. 

Understanding the interaction of interstitials and impurities with the motion of
dislocations through a crystal is a central goal of theoretical metallurgy with
with work on this subject beginning in the 1950s \cite{cochardt55}.

The local density of states for atoms at a screw dislocation core have been computed by \cite{paidar81,masuda81}. 
The change in the local bonding picture induced by a vacancy in Fe was examined in Ref.~\cite{masuda82,ohta87}.

\section{Grain Boundaries}
The analysis of Read and Shockley in Ref.~\cite{read50} serves as a comprehensive
introduction to the field of interfaces in condensed matter. The definitive textbook
on the subject of interfaces in materials is Sutton and Baluffi Ref.~\cite{sutton95}.

A number of experimental probes are capable of detecting the presence of 
grain boundaries but efficient and accurate means of determining 
the orientation relationships between grains remains an outstanding problem.
For an analysis of the statistic arrangement of grains using quaternion
algebra see Ref.~\cite{sutton96}.

The possibility of examining grain boundaries atomistically has interested
researchers since it became practical to perform atomistic simulations
with realistic potentials see Refs.~\cite{bristowe75,wolf83,paxton87,paxton88,paxtonsutton88,
kohyama88,kohyama94,paxton96,rittner96,tschopp07,momida13,du11,du12,mceniry18} for a representative
snapshot of the way these studies have evolved over time.


\subsection{Discrete Boundary Dislocation Networks}
For isotropic elasticity we can gain some useful practice in contour integration and arrive
at a closed set of expressions to obtain. These expressions are given in Ref.~\cite{sutton95}.



\section{Alloys}
\cite{glaser81}
Lattice periodicity allows for the use of Bloch's theorem. The Fourier transform
is well defined and the eigenstates can be grouped according the their Bloch vector.
For alloys its quite likely there will not be an identifiable periodicity in 
the structure.
%
\begin{equation}
H_{nm} = e_{n}\delta_{nm} + V_{nm}
\end{equation}
%
Mookerjee and Haydock extended the recursion method to the case of random alloys.
The developement of the recursion method to handle random material systems is 
itself quite ingenious \cite{mookerjee, haydock74}. The Hamiltonian is extended
so that tight binding hops on a real lattice are made along with potential hops
on a "field" lattice. The field lattice can be cast in tridiagonal form to represent
the underlying statistical distribution of the random alloy. Different graphs are available
to represent rectangular random distributions, binary distributions, Gaussian distributions
etc. Fig.~\ref{fig:fieldsite.png} gives an example of the first few hops on a "field-site" 
lattice to obtain the averaged value of a Green's function. The hopping of the recursion method
is now coupling lattice sites and taking into account the underlying distribution of 
the underlying parameters.

Despite its elegance the number of paths now emanating from an initial orbital, both
for electrons to hop from orbital to orbital (lattice hops) and for the hops between 
statistical configurations (field hops) are considerable and simplifying approximations 
need to be found due to the difficulty of enumerating all the possible paths.

Early work on DOS of alloys \cite{cubiotti77}. 

\section{Metallic Surfaces}
Elegant experimentation can access some surprisingly delicate quantities \cite{whipp34}.
See \cite{bullet77} for calculations of the adsorption energy of H on W and Pt.
Hydrogen on metal \cite{garciavidal91}.

\section{Conclusion}
In this chapter we have looked at a number of the important concepts
in metallurgy. The motion of dislocations, 
the structure of grain boundaries, and alloying. 

%Masuda-Jindo ISIJ 28, 843 (1988)
%Dilute Alloys
%J Friedel Adv. Phys. 3, 446 (1954)
%Non-Hermitian Matrices
%R. Haydock and M.J. Kelly. J. PHys. C 8, L290
%R. Haydock, J. Phys. A 7, 2120 (1974)
%Lave Phases:
%A. K. Sinha, Prog. Mater. Sci. 15, 81 (1972)
%Anderson Localization
%R. Haydock Philos. Mag. [Part] B 37, 97 (1978)
%R. Haydock and A. Mookerejee, J. Phys. C 7, 3001 (1974)
%UV Photoemission
%R. K. C. McLean and R. Haydock, J. Phys. C 10, 1929 (1977)
%https://materialscience.uoregon.edu/wp-content/uploads/2016/02/tightbind.pdf
%Self-Consistent Tight Binding Orbitals
%Philos. Mag [8] 35, 845 (1977) #
%Binary Alloys 
%R. L. Jacobs, J. Phys. F 3, 933 (1973)
%and stacking in La \cite{duthie77}.

%Atomic stress tensors:
%Sutton 88
%Nielsen O H and Martin R M 1983 Phys. Rev. Lett. 50 697
%Nielsen O H and Martin R M 1985a Phys. Rev. B 32 3780
%-                          1985b Phys. Rev. B 32 3792
