\title{Metallic Cohesion}
  Cottrell noted that metallurgy would require a "theory... which explains
the properties of metallic solids and the explanation of the origin of this
structure from the strutures of individual atoms." 

In a similar vein Heine noted:
"What we lack is a fundamental quantum theory of the cohesion and structure
of a wide diversity of solids." 

One of the main thrusts of this, increasingly misnamed note, is that the route
to a comprehensive framework for a quantum theory of cohesion must come
from a mathematically rigorous description of electronic structure based
on the local atomic environment. 

\section{Pseudopotential Theory}
  Each element in its atomic form is identical to every element of the same kind,
yet it is difficult to formulate, mathematically, a description applicable 
to that same atom in all its potentially different environments. 

  The properties of an atom as it stands alone, perhaps sitting in a 
vacuum, is in some ways completely different from the same atom 
surrounded by copies of itself, and there are changes again
in the collective properties of the material when that atom is 
surrounded by differing neighbour atoms.

  Electronic structure is a theory of the local environment, the effects depend on nearest
neighbours and swapping atoms in and out of a crystal matrix will 
effect local properties but not general ones. 

However it is also known that tiny variations in the material composition 
and microstructure can in some cases
produce dramatic effects across macroscopic length scales completely 
altering the properties of the material.

This apparent paradox can be resolve by refining the definitions of local
and tiny variations. In fact quantifying the scale of these interactions 
is something the recursion method is particularly good at.

\section{Single Particle Recursion: PDOS Lave Phases}

\section{Many Body Recursion: PDOT Calculations of Cohesive Properties}
  For a hubbard U model of cohesion energies in FCC and HCP metal using the Project density of transitions
\cite{haydock14}. For an application to Heisenberg chains where electrons are fixed and can only exchange
spin with neighbours see \cite{haydock00}.
