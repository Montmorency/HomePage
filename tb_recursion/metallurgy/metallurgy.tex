\chapter{Metallurgy}
\label{chap:metallurgy}
\section{Introduction}
  Cottrell noted that metallurgy would require a "theory... which explains
the properties of metallic solids and the explanation of the origin of this
structure from the structures of individual atoms." 

This chapter will examine the ability to calculate the electronic and mechanical
properties of transition metals and their alloys. The sorts of quantities
of interest are elastic constants, dislocation core structure, grain boundaries
and finally the extent of bond mobility in metals in the presence of impurity
segregants. The final part treats the problem of embrittlement which is of considerable
interest in the metallurgical community.

%\section{Many Body Recursion: Cohesive Properties}
%Similarly Heine noted "What we lack is a fundamental quantum 
%theory of the cohesion and structure of a wide diversity of solids." 
%For a hubbard U model of cohesion energies in FCC and HCP metal using the Project density of transitions
%\cite{haydock14}.

\section{Transition Metals: Fe}
If we seek a quantum theory of metallurgy Fe is the place to start.
Local descriptions of the electronic description of iron may take two forms. 

The tight binding parameters determined by pettifor, and the Localized 
Wannier functions studied by \cite{yates06}. 

The parameters they use are as follows 16x16x16 0.02 Ry fermi smearing, obtaining
a spin magnetic moment of 2.22 $\mu_{b}$. Norm Conserving pseudopotential with KE cutoff
of 60 Hartree (120 Ry.)

The self consistent potential is then frozen and a uniform nxnxn grid of k points is
used to compute the Wannier functions. 18 WFs are targeted (s,p and d gives a 9x9 matrix
and then the two two spins doubles the size), with spin up and 
down. 28 bands calculated at each k point and any bands above 58 eV are excluded.

In the course of the Wannierization, the initial guess was using s, p, and (eg and t2g)
type d-like Gaussians of pure spin character transformed the bands into the t2g like 
Wannier functions and the $e_{g}$, s and p-like states formed six hybrid 
$sp^{3}d^{2}$-type orbitals. Thus with the initial choice of 6 $sp^{3}d^{2}$ and 3 
$t_{2g}$  type orbitals a 100 iterations reach a neatly converged spread functional.

The $sp^{3}d^{2}$ orbitals are directed along $\pm x, \pm y, \pm z$ axis 
slightly shifted off the center of the atoms, the $xy$, $xz$ $yz$ orbitals are
$t_{2g}$ like and centered on the atoms. An ab initio grid of 8x8x8 was sufficient
to obtain well localized Wannier functions.

\section{Recursion for Dislocations}
For the BCC Metals \cite{terakura84} used the recursion method to compute the elastic constants.
The local density of states for atoms at a screw dislocation core have been computed by \cite{paidar81}. 

\section{Grain Boundaries}
Dislocation core structure and stress Field of grain boundaries.

\section{Alloys}
%GrainBoundaries 
%Masuda-Jindo ISIJ 28, 843 (1988)
\section{Random Alloys}
Lattice periodicity allows for the use of Bloch's theorem. The Fourier transform
is well defined and the eigenstates can be grouped according the their Bloch vector.
%
\begin{equation}
H_{nm} = e_{n}\delta_{nm} + V_{nm}
\end{equation}
%
It is also possible to consider the Fourier transform from a Bayesian perspective
where we are picking out features of a distribution with varying amounts of noise.

Mookerjee and Haydock extended the recursion method to the case of random alloys.
The developement of the recursion method to handle random material systems is 
itself quite ingenious \cite{mookerjee , haydock74}. The Hamiltonian is extended
so that tight binding hops on a real lattice are made along with potential hops
on a "field" lattice. The field lattice can be cast in tridiagonal form to represent
the underlying statistical distribution of the random alloy. Different graphs are available
to represent rectangular random distributions, binary distributions, Gaussian distributions
etc. Fig.~\ref{fig:fieldsite.png} gives an example of the first few hops on a "field-site" 
lattice to obtain the averaged value of a Green's function. The hopping of the recursion method
is now coupling lattice sites and taking into account the underlying distribution of 
the underlying parameters.

Despite its elegance the number of paths now emanating from an initial orbital, both
for electrons to hop from orbital to orbital (lattice hops) and for the hops between 
statistical configurations (field hops) are considerable and simplifying approximations 
need to be found due to the difficulty of enumerating all the possible paths.

Early work on DOS of alloys \cite{cubiotti77}. 

\section{Embrittlement of Grain Boundaries}
In this section we describe a proposed mechanism of embrittlement via impurity
segregation \cite{haydock82}.

\section{Conclusion}
In this chapter we have looked at dislocations, grainboundaries, random alloys, and the embrittlement
phenomenon.

