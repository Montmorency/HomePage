\chapter{Metallurgy}
  Cottrell noted that metallurgy would require a "theory... which explains
the properties of metallic solids and the explanation of the origin of this
structure from the structures of individual atoms." 

Similarly Heine noted "What we lack is a fundamental quantum 
theory of the cohesion and structure of a wide diversity of solids." 

One of the main thrusts of this, increasingly misnamed note, is that the route
to a comprehensive framework for a quantum theory of cohesion must come
from a mathematically rigorous description of electronic structure based
on the local atomic environment. 

\section{Single Particle Recursion: Lave Phases}

\section{Many Body Recursion: Cohesive Properties}
For a hubbard U model of cohesion energies in FCC and HCP metal using the Project density of transitions
\cite{haydock14}.

\section{Random Alloys}
Lattice periodicity allows for the use of Bloch's theorem. The Fourier transform
is well defined and the eigenstates can be grouped according the their Bloch vector.
%
\begin{equation}
H_{nm} = e_{n}\delta_{nm} + V_{nm}
\end{equation}
%
It is also possible to consider the Fourier transform from a Bayesian perspective
where we are picking out features of a distribution with varying amounts of noise.

Mookerjee and Haydock extended the recursion method to the case of random alloys.
The developement of the recursion method to handle random material systems is 
itself quite ingenious \cite{mookerjee , haydock74}. The Hamiltonian is extended
so that tight binding hops on a real lattice are made along with potential hops
on a "field" lattice. The field lattice can be cast in tridiagonal form to represent
the underlying statistical distribution of the random alloy. Different graphs are available
to represent rectangular random distributions, binary distributions, Gaussian distributions
etc. Fig.~\ref{fig:fieldsite.png} gives an example of the first few hops on a "field-site" 
lattice to obtain the averaged value of a Green's function. The hopping of the recursion method
is now coupling lattice sites and taking into account the underlying distribution of 
the underlying parameters.
%
%\begin{figure}
%\caption{Field-Site lattice to directly compute the average quantities in a tight binding formalism}
%\end{figure}
%
Despite its elegance the number of paths now emanating from an initial orbital are
considerable and simplifying approximations need to be found due to the difficulty of 
enumerating all the possible paths through the lattice and the along the field vectors.

\section{Embrittlement of Metallic Surfaces}
Haydock has also introduced a mechanism analogous to the polaron model for describing the
"covalentization" of metallic bonds that is a potential mechanism for the embrittlement
of metals \cite{haydock82}.

\section{Mechanics}
For a study of the effect of impurity concentration on the velocity of dislocations in semiconductors see \cite{frisch67}.
